\documentclass{beamer}
\usepackage{up}

\title{Указатели и псевдоними}

\date{7 декември 2016 г.}

\begin{document}

\begin{frame}
  \titlepage
\end{frame}

\section{Указатели}

\begin{frame}
  \frametitle{Тип указател}
  
  \begin{itemize}
  \item \textbf{Множество от стойности:} всички възможни lvalue от даден тип и специалната стойност \tt{NULL}.
  \item Интегрален \alert{нечислов} тип
  \item Параметризиран тип: ако \tt T е тип данни, то \tt{T*} е тип ``указател към елемент от тип \tt T''
  \item Физическо представяне: цяло число, указващо адреса на указваната lvalue в паметта
  \item Стойностите от тип ``указател'' са с размера на машинната дума
    \begin{itemize}
    \item 32 бита (4 байта) за 32-битови процесорни архитектури
    \item 64 бита (8 байта) за 64-битови процесорни архитектури
    \end{itemize}
  \end{itemize}
\end{frame}

\begin{frame}
  \frametitle{Операции с указатели}

  \begin{itemize}
  \item рефериране (\tta\&<lvalue>)
  \item дерефериране (\tta*<указател>)
    \begin{itemize}
    \item \alert{унарна операция!}
    \end{itemize}
  \item сравнение (\tt{==}, \tt{!=}, \tt<, \tt>, \tt{<=}, \tt{>=})
  \item указателна аритметика (\tt+,\tt-,\tt{+=},\tt{-=},\tt{++},\tt{--})
  \item извеждане (\tt{<{}<})
  \item \alert{няма извеждане! (\tt{>{}>})}
  \end{itemize}
\end{frame}

\tikzset{cell/.style={rectangle,minimum size=3ex,draw,fill=diagramblue}}

\begin{frame}
  \frametitle{Дефиниране на указателни променливи}
  
  <тип> \tta*<име> [ \tta= <израз> ] \{\tta, \tta*<име> [ \tta= <израз> ] \}\tta;\\[1em]
  \begin{overlayarea}{\textwidth}{10em}
  \textbf{Примери:}
  \pause
  \begin{columns}[T,onlytextwidth]
    \begin{column}{.5\textwidth}
      \begin{itemize}[<+->]
      \item \lst{int *pi;}
      \item \lst{double *pd = NULL;}
      \item \lst{double d = 1.23;}
      \item \lst{double *qd = \&d;}
      \item \lst{double **qqd = \&qd;}
      \end{itemize}      
    \end{column}
    \begin{column}{.5\textwidth}
      \begin{onlyenv}<2->
        \begin{tikzpicture}
          \node[label={\tt{pi}},cell] (pi) {};
          \draw[->,>=stealth] (c.center)
          .. controls +(4ex,1ex) and +(1ex,4ex) .. ++(6ex,0)
          .. controls +(0,-1ex) and +(2ex,-2ex) .. +(4ex,0);
        \end{tikzpicture}
      \end{onlyenv}
      \begin{onlyenv}<3->
        \begin{tikzpicture}
          \node[label={\tt{pd}},cell] (pd) {};
          \draw (pd.south west) -- (pd.north east);
        \end{tikzpicture}
      \end{onlyenv}\\[2em]
      \begin{onlyenv}<4>
        \begin{tikzpicture}
          \node[label={\tt{d}},cell] (d) {\tt{1.23}};
        \end{tikzpicture}
      \end{onlyenv}
      \begin{onlyenv}<5>
        \begin{tikzpicture}
          \node[label={\tt{d}},cell] (d) {\tt{1.23}};
          \node[label={\tt{qd}},cell,right=of d] (qd) {};
          \draw[->,>=stealth] (qd.center) -> (d);
        \end{tikzpicture}
      \end{onlyenv}
      \begin{onlyenv}<6>
        \begin{tikzpicture}
          \node[label={\tt{d}},cell] (d) {\tt{1.23}};
          \node[label={\tt{qd}},cell,right=of d] (qd) {};
          \node[label={\tt{qqd}},cell,right=of qd] (qqd) {};
          \draw[->,>=stealth] (qd.center) -> (d);
          \draw[->,>=stealth] (qqd.center) -> (qd);
        \end{tikzpicture}
      \end{onlyenv}
    \end{column}
  \end{columns}
  \end{overlayarea}
\end{frame}

\begin{frame}
  \frametitle{Рефериране и дерефериране}
  
  \begin{itemize}[<+->]
  \item \tta{\&}<име> --- указател към променливата <име>
  \item \tta*<указател> --- мястото в паметта, сочено от <указател>
  \item \textbf{Примери:}
    \begin{itemize}
    \item \lst{int x = 5, *p = \&x;}
    \item \lst{int *q = p, y = *p + 2;}
    \item \lst{*p++; p = \&y;}
    \item \lst{*q = 1; *p = *q;}
    \end{itemize}
  \end{itemize}
\end{frame}

\end{document}
