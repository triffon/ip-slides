\documentclass[alsotrans]{beamerswitch}
\usepackage{iprog}

\title{Указатели и препратки}

\date{6--13 декември 2021 г.}

\tikzset{label anchor/.style={tikz@label@post/.append style={anchor=#1}}}
\tikzset{every label/.style={
    text depth=.2em,
    text height=0.8em,
    label position=north west,
    label anchor=south west
  }
}

\tikzset{cell/.style={
    rectangle,
    minimum size=3ex,
    draw,
    fill=diagramblue
  }
}

\newcommand{\thedouble}{
  \node[label={\tt{d}},cell] (d) {\tt{1.23}};
}

\newcommand{\thepointertodouble}{
  \thedouble
  \node[label={\tt{qd}},cell,right=of d] (qd) {};
  \draw[->,>=stealth] (qd.center) -> (d);
}

\newcommand{\thepointertopointertodouble}{
  \thepointertodouble
  \node[label={\tt{qqd}},cell,right=of qd] (qqd) {};
  \draw[->,>=stealth] (qqd.center) -> (qd);
}


\tikzset{
  n/.style={
    rectangle,
    minimum width=4ex,
    minimum height=1em,
    text height=0.75em,
    text depth=0.25em,
    fill=diagramblue,
    draw},
  m/.style={
    matrix of nodes,
    ampersand replacement=\&,
    nodes=n
  }
}

\newcommand{\thearray}[1]{
  \matrix (a) [m] {
    \tt{#1} \& \tt{a[1]} \& \tt{a[2]} \& \tt{a[3]} \& \tt{a[4]} \\
  };
  \node (pa) [left=of a-1-1] {\tt{a}};
  \draw[->,>=stealth] (pa) -> (a-1-1);
}

\newcommand{\thepointer}{
  \node (p) [below=of a-1-1,label={[label anchor=east]left:{\tt{p}}},n] {};
  \draw[->,>=stealth] (p) -> (a-1-1);
}

\newcommand{\thestringconstant}{
  \node (p) [cell,label={\tt{p}}] {};
  \node (pv) [cell,right=of p] {\tt{Hello, world!}};
}

\newcommand{\thestring}[1]{
  \node (qv) [cell,below=of pv] {\tt{H#1 C++!}};
  \node (q) [left=of qv,label={\tt{q}}] {};
  \draw[->,>=stealth] (q) -> (qv);
}

\newcommand{\thefirstedge}{
  \draw[->,>=stealth,very thick] (p.center) -> (pv);
}

\newcommand{\thesecondedge}{
  \draw[->,>=stealth,very thick] (p.center) -> (qv.north west);
}


\newcommand{\labeledcell}[2]{
  \node [cell,label={\tt{#1}}] {\tt{#2}};
}

\begin{document}

\begin{frame}
  \titlepage
\end{frame}

\section{Указатели}

\begin{frame}
  \frametitle{Тип указател}

  \begin{itemize}[<+->]
  \item \textbf{Множество от стойности:} всички възможни lvalue от даден тип и специалната стойност \lst{nullptr}.
  \item Интегрален \alert{нечислов} тип
  \item Параметризиран тип: ако \tt T е тип данни, то \tt{T*} е тип ``указател към елемент от тип \tt T''
  \item Физическо представяне: цяло число, указващо адреса на указваната lvalue в паметта
  \item Стойностите от тип ``указател'' са с размера на машинната дума
    \begin{itemize}
    \item 32 бита (4 байта) за 32-битови процесорни архитектури
    \item 64 бита (8 байта) за 64-битови процесорни архитектури
    \end{itemize}
  \end{itemize}
\end{frame}

\begin{frame}
  \frametitle{Операции с указатели}

  \begin{itemize}[<+->]
   \item рефериране (\tta\&<lvalue>)
   \item дерефериране (\tta*<указател>)
     \begin{itemize}[<.->]
     \item \alert{унарна операция!}
     \end{itemize}
   \item сравнение (\tt{==}, \tt{!=}, \tt{<}, \tt{>}, \tt{<=}, \tt{>=})
   \item указателна аритметика (\tt+,\tt-,\tt{+=},\tt{-=},\tt{++},\tt{-{}-})
  \item извеждане (\tt{<{}<})
  \item \alert{няма въвеждане! (\tt{>{}>})}
  \end{itemize}
\end{frame}

\begin{frame}
  \frametitle{Дефиниране на указателни променливи}

  <тип> \tta*<име> [ \tta= <израз> ] \{\tta, \tta*<име> [ \tta= <израз> ] \}\tta;\\[2ex]
  \begin{overlayarea}{\textwidth}{23ex}
  \textbf{Примери:}
  \pause
  \begin{columns}[T,onlytextwidth]
    \begin{column}{.5\textwidth}
      \begin{itemize}[<+->]
      \item \lst{int *pi;}
      \item \lst{double *pd = nullptr;}
      \item \lst{double d = 1.23;}
      \item \lst{double *qd = \&d;}
      \item \lst{double **qqd = \&qd;}
      \end{itemize}
    \end{column}
    \begin{column}{.5\textwidth}
      \begin{onlyenv}<2->
        \begin{tikzpicture}
          \node[label={\tt{pi}},cell] (pi) {};
          \draw[->,>=stealth] (pi.center)
          .. controls +(4ex,1ex) and +(1ex,4ex) .. ++(6ex,0)
          .. controls +(0,-1ex) and +(2ex,-2ex) .. +(4ex,0);
        \end{tikzpicture}
      \end{onlyenv}
      \begin{onlyenv}<3->
        \begin{tikzpicture}
          \node[label={\tt{pd}},cell] (pd) {};
          \draw (pd.south west) -- (pd.north east);
        \end{tikzpicture}\\[4ex]
      \end{onlyenv}
      \only<4| trans:0>{\tikz{\thedouble}}
      \only<5| trans:0>{\tikz{\thepointertodouble}}
      \only<6>{\tikz{\thepointertopointertodouble}}
    \end{column}
  \end{columns}
  \end{overlayarea}
\end{frame}

\begin{frame}
  \frametitle{Рефериране и дерефериране}

  \begin{itemize}[<+->]
  \item \tta{\&}<име> --- указател към променливата <име>
  \item \tta*<указател> --- мястото в паметта, сочено от <указател>
  \item \textbf{Примери:}
    \begin{itemize}
    \item \lst{int x = 5, *p = \&x;}
    \item \lst{int *q = p, y = *p + 2;}
    \item \lst{(*p)++; p = \&y;}
    \item \lst{*q = 1; *p = *q;}
    \end{itemize}
  \item \tt{\&}<lvalue> връща като резултат <rvalue>!
    \begin{itemize}
    \item \sta{\&3}
    \item \sta{\&x = p;}
    \end{itemize}
  \item \tt*<rvalue> връща като резултат <lvalue>!
    \begin{itemize}
    \item \tt{*p = x;}
    \item \tt{**qqd = 3.15;}
    \end{itemize}
  \item операциите са дуални една на друга и се унищожават взаимно
    \begin{itemize}
    \item \tt{\&(*p)} \eqv \tt{p}
    \item \tt{*(\&x)} \eqv \tt{x}
    \end{itemize}
  \end{itemize}
\end{frame}

\section{Указателна аритметика}

\begin{frame}[fragile]
  \frametitle{Указатели и масиви}
  В C++ има много тясна връзка между указатели и масиви.
  \begin{afact}
    Името на масив е \alert{константен указател} към първия му елемент.
  \end{afact}
  \pause
  \textbf{Примери:}
  \begin{columns}[T,onlytextwidth]
    \begin{column}{.4\textwidth}
      \begin{itemize}[<+->]
      \item \lst{int a[5];}
      \item \lst{int* p = a;}
      \item \lst{*p = 15;}
      \item \lst{cout << a[0];}
      \item \lst{*a = 20;}\sta{a = p;}
      \end{itemize}
    \end{column}
    \begin{column}{.6\textwidth}
      \only<2| trans:0>{\tikz{\thearray{a[0]}}}
      \only<3-4| trans:0>{\tikz{\thearray{a[0]}\thepointer}}
      \only<5-6| trans:0>{\tikz{\thearray{ 15 }\thepointer}}
      \only<7>{\tikz{\thearray{ 20 }\thepointer}}
    \end{column}
  \end{columns}
\end{frame}

\begin{frame}
  \frametitle{Указателна аритметика}

  \begin{itemize}[<+->]
  \item Указателната аритметика позволява по дадена отправна точка в паметта (указател) да реферираме съседни на нея клетки.
  \item За целта трябва да укажем колко клетки напред или назад в паметта искаме да прескочим.
  \item Синтаксис:
    \begin{itemize}
    \item{} <указател> [\tta+ | \tta-] <цяло\_число>
    \item{} <цяло\_число> \tta+ <указател>
    \end{itemize}
  \item прескачаме <цяло\_число> клетки напред (\tta+) или назад (\tta-) от адреса, сочен от <указател>
  \end{itemize}
\end{frame}

\begin{frame}
  \frametitle{Големина на тип}

  \begin{itemize}[<+->]
  \item Но... какво означава ``прескачаме n клетки''?
  \item \alert{Зависи от типа, който указваме!}
    \begin{itemize}
    \item \tt{p + 2} означава ``прескочи 2 байта напред'', ако
      \lst{char* p;}
    \item \tt{p + 2} означава ``прескочи 8 байта напред'', ако \lst{int*
        p;}
    \item \tt{p + 2} означава ``прескочи 16 байта напред'', ако
      \lst{double* p;}
    \end{itemize}
  \item \tta{sizeof(}<тип>|<израз>\tta) --- размера в байтове, заеман
    в паметта от <израз> или от стойност от <тип>
  \item Така, ако имаме \lst{T* p;}...
  \item ...тогава \lst{p + i} прескача \tt{i * sizeof(T)} байта напред
  \item \lst{(long)p} --- цялото число, съответстващо на адреса сочен от
    \tt{p}
  \item \lst{p + i} \eqv \lst{(T*)((long)p + i * sizeof(T))}
  \end{itemize}
\end{frame}

\begin{frame}
  \frametitle{Указателна аритметика за масиви}

  \begin{afact}
    Името на масив е \alert{константен указател} към първия му елемент.\\
    \pause
    Освен това, \tt{a[i]} \eqv \tt{*(a + i)}
  \end{afact}
  \pause
  \textbf{Примери:}
  \begin{itemize}[<+->]
  \item \lst{int a[5], x;}\hspace{4em}\tikz{\thearray{a[0]}}
  \item \lst{cout << *a;}
  \item \lst{*(a + 1) = 7;}
  \item \lst{--*(a + 4);}
  \item \sta{a++; a-{}-; a = \&x;}
  \item Странно, но вярно: \tt{a[i]} \eqv \tt{*(a+i)} \eqv \tt{*(i+a)} \eqv \tt{i[a]}
  \end{itemize}
\end{frame}

\begin{frame}[fragile]
  \frametitle{Указатели и низове}

  Низовете са масиви от символи\\[2ex]
  \begin{tikzpicture}
    \matrix (a) [m] {
      \tt{H} \&\tt{e} \&\tt{l} \&\tt{l} \&\tt{o} \&\tt{,} \&\tt{ } \&\tt{w} \&\tt{o} \&\tt{r} \&\tt{l} \&\tt{d} \&\tt{!} \& \tt{\textbackslash 0} \\
    };
    \node (s) [left=of a] {s};
    \draw[->,>=stealth] (s) -> (a-1-1);
    \node (p) [n,below=of a-1-12,label={[label anchor=east]left:{\tt{p}}}] {};
    \draw[->,>=stealth] (p) -> (a-1-12);
  \end{tikzpicture}
  \pause
  \textbf{Примери:}
  \begin{columns}[T,onlytextwidth]
    \begin{column}{.55\textwidth}
\begin{lstlisting}
char s[] = "Hello, world!";
@\pause@
void print(char* p) {
  while (*p) cout << *p++;
}
\end{lstlisting}
    \end{column}
    \pause
    \begin{column}{.45\textwidth}
\begin{lstlisting}
int strlen(char* p) {
  int n = 0;
  while (*p++) n++;
  return n;
}
\end{lstlisting}
    \end{column}
  \end{columns}
\end{frame}

\section{Указатели и константи}

\begin{frame}[fragile]
  \frametitle{Указатели и константи}

  \begin{itemize}[<+->]
  \item Константен указател (който е константа)
    \begin{itemize}
    \item{} <тип>\tta{* const} \hspace{12em}
      \begin{tikzpicture}[>=stealth]
        \graph { ""[very thick,cell] -> ""[cell] };
      \end{tikzpicture}
    \item \lst{int x, *p = &x;}
    \item \lst{int* const q = p; }\sta{q = p + 2;}\lst{ *q = 5;}
    \end{itemize}
  \item Указател към константа (сочещ към константа)
    \begin{itemize}
    \item{} \tta{const} <тип>\tta* \eqv <тип> \tta{const*} \hspace{2em}
      \begin{tikzpicture}[>=stealth]
        \graph { ""[cell] ->[very thick] ""[cell] };
      \end{tikzpicture}
    \item \lst{int x, *p = &x;}
    \item \lst{int const* q = &x; q++; }\sta{p = q; *q = 5;}
    \end{itemize}
  \item \alert{Ако \tt p е указател към константа, то \tt{*p} е <rvalue>}
  \item \alert{Ако \tt x е константа, то \tt{\&x} е указател към константа}
  \end{itemize}
\end{frame}

\begin{frame}[fragile]
  \frametitle{Указатели и низови константи}
    \small

  \begin{afact}
    Името на низ е \alert{константен указател} към първия му символ (\lst{char* const}).\\\pause
    Низовите константи са \alert{указатели към константен символ} (\lst{char const*}).
  \end{afact}
  \pause
  \begin{columns}[T,onlytextwidth]
    \begin{column}{.6\textwidth}
      \begin{itemize}[<+->]
      \item \lst{char const* p = "Hello, world!";}
      \item \sta{char* q = \quot Hi C++!\quot;}
      \item \lst{char q[] = "Hi C++!";}
      \item \lst{p = q;}
      \item \lst{q[1] = 'o'; }\sta{p[1] = 'o';}
      \item \lst{cout << p[4];}
      \end{itemize}
    \end{column}
    \begin{column}{.4\textwidth}
      \only<3-4| trans:0>{\tikz{\thestringconstant\thefirstedge}}
      \only<5| trans:0>{\tikz{\thestringconstant\thefirstedge\thestring{i}}}
      \only<6| trans:0>{\tikz{\thestringconstant\thestring{i}\thesecondedge}}
      \only<7->{\tikz{\thestringconstant\thestring{o}\thesecondedge}}
    \end{column}
  \end{columns}
\end{frame}

\begin{frame}
  \frametitle{Указатели и многомерни масиви}

  \small
  \begin{columns}[T,onlytextwidth]
    \begin{column}{.48\textwidth}
      \begin{itemize}[<+->]
        \setlength{\itemindent}{-1em}
      \item \lst{int a[2][2][3];}
      \item \tt a е от тип \lst{int (*const)[2][3];}
      \item \tt{a[i]} е от тип \lst{int (*const)[3];}
      \item \tt{a[i][j]} е от тип \lst{int *const;}
      \end{itemize}
    \end{column}
    \begin{column}{.52\textwidth}
      \begin{itemize}[<+->]
        \setlength{\itemindent}{-1em}
      \item \tt{a[i] \eqv *(a+i)}
      \item \tt{a[i][j] \eqv *(*(a+i)+j)}
      \item \tt{a[i][j][k] \eqv *(*(*(a+i)+j)+k)}
      \item \tt{a[1][1][1] \eqv *(*(*(a+1)+1)+1)}
      \end{itemize}
    \end{column}
  \end{columns}
  \vspace{2em}
  \begin{center}
    \scalebox{.95}{
      \begin{tabular}{|*{12}{@{\hspace{0.1em}}c@{\hspace{0.1em}}|}}
        \hline
        \multicolumn{12}{|c|}{\tt{a}}\\
        \hline
        \multicolumn{6}{|c|}{\tt{a[0]}}&\multicolumn{6}{|c|}{\tt{a[1]}}\\
        \hline
        \multicolumn{3}{|c|}{\tt{a[0][0]}}&\multicolumn{3}{|c|}{\tt{a[0][1]}}&\multicolumn{3}{|c|}{\tt{a[1][0]}}&\multicolumn{3}{|c|}{\tt{a[1][1]}}\\
        \hline
        \tiny a[0][0][0]&\tiny a[0][0][1]&\tiny a[0][0][2]&\tiny a[0][1][0]&\tiny a[0][1][1]&\tiny a[0][1][2]&\tiny a[1][0][0]&\tiny a[1][0][1]&\tiny a[1][0][2]&\tiny a[1][1][0]&\tiny a[1][1][1]&\tiny a[1][1][2]\\
        \hline
      \end{tabular}}
  \end{center}
\end{frame}

\begin{frame}
  \frametitle{Указател към неизвестен тип}

  \begin{itemize}[<+->]
  \item \textbf{Проблем:} Не можем да насочваме един и същ указател към променливи от различен тип!
  \item \textbf{Решение:} \tta{void*} --- указател към неизвестен тип
  \item[$\checkmark$] Преобразуваме автоматично от \tt{T*} към \tt{void*}
    \begin{itemize}
    \item \lst{int x, *p; void *q = p;}
    \item \lst{void *r = &x, *pr = &r, *s = &r;}
    \end{itemize}
  \item[$\times$] \alert{Няма} автоматично преобразуване от \tt{void*} към \tt{T*}
    \begin{itemize}
    \item \lst{int* p; void* q = p;}
    \item \sta{int* r = q;}
    \item \lst{int* s = (int*)q;}
    \end{itemize}
  \item[$\times$] \alert{Няма} дерефериране (\tt{void} е празният тип)
    \begin{itemize}
    \item \lst{int x; void* p = &x;}
    \item \sta{*p = 2; void y = *p;}
    \end{itemize}
  \end{itemize}
\end{frame}

\section{Препратки}

\begin{frame}
  \frametitle{Препратка (reference)}

  \begin{itemize}[<+->]
  \item \textbf{Множество от стойности:} всички възможни lvalue от даден тип
  \item Параметризиран тип: ако \tt T е тип данни, то \tt{T\&} е тип ``препратка към елемент от тип \tt T''
  \item Физическо представяне:
    \begin{itemize}
    \item на теория: както реши компилаторът
    \item на практика: екиквалентно на \alert{константен указател към T}
    \item \tt{T\& \eqv T* const}
    \end{itemize}
  \end{itemize}
\end{frame}

\begin{frame}
  \frametitle{Дефиниране на препратка}

  \begin{itemize}[<+->]
  \item{} <тип>\tta{\&} <идентификатор> \tta= <обект>\\
    \hspace{2.5em} \{\tta{, \&}<идентификатор> \tta= <обект> \}\tta;
  \item инициализацията е \alert{задължителна}!
    \begin{itemize}
    \item както и на константните указатели
    \end{itemize}
  \item препратката \alert{не може} да се пренасочва към друг обект
    \begin{itemize}
    \item както и константните указатели
    \end{itemize}
  \end{itemize}
  \onslide<+->
  \textbf{Пример:}
  \begin{columns}[T,onlytextwidth]
    \begin{column}{.5\textwidth}
      \begin{itemize}[<+->]
      \item \lst{int x = 3;}
      \item \lst{int &a = x, b = a;}
      \item \lst{int &c = b;}
      \item \lst{a = c + 5;}
      \end{itemize}
    \end{column}
    \begin{column}{.5\textwidth}
      \only<7| trans:0>{\tikz{\labeledcell{x}3}}%
      \only<8-10| trans:0>{\tikz{\labeledcell{x,a}3}}%
      \only<11>{\tikz{\labeledcell{x,a}8}}%
      \hspace{2em}%
      \only<8| trans:0>{\tikz{\labeledcell{b}3}}%
      \only<9->{\tikz{\labeledcell{b,c}3}}
    \end{column}
  \end{columns}
\end{frame}

\begin{frame}
  \frametitle{Сравнение на препратки и указатели}

  \begin{itemize}[<+->]
  \item Самата препратка не е обект, който може да бъде манипулиран
    \begin{itemize}
    \item Указателите могат да бъдат насочвани към различни адреси
    \end{itemize}
  \item Препратките винаги са закачени за съществуващ обект
    \begin{itemize}
    \item Указателите могат да имат стойност \lst{nullptr}
    \end{itemize}
  \item Препратката не се различава от оригинала
    \begin{itemize}
    \item Указателят изисква изрично дерефериране с \tt*
    \end{itemize}
  \item Препратките на един и същ обект са взаимозаменяеми
  \end{itemize}
\end{frame}

\begin{frame}
  \frametitle{Константни препратки}

  \begin{itemize}[<+->]
  \item \tta{const }<тип>\tta\& \eqv <тип>\tta{ const\&}
  \item Представляват константен ``изглед'' на дадено място в паметта
  \item Няма разлика между константи препратки и препратки на константи
  \end{itemize}
  \onslide<+->
  \textbf{Пример:}
  \begin{columns}[T,onlytextwidth]
    \begin{column}{.5\textwidth}
      \begin{itemize}[<+->]
      \item \lst{int a = 3;}
      \item \lst{a++;}
      \item \lst{int\& b = a;}
      \item \lst{b++;}
      \item \lst{int const& c = b;}
      \item \sta{c++;}
      \end{itemize}
    \end{column}
    \begin{column}{.5\textwidth}
      \only<5| trans:0>{\tikz{\labeledcell a3}}
      \only<6| trans:0>{\tikz{\labeledcell a4}}
      \only<7| trans:0>{\tikz{\labeledcell{a,b}4}}
      \only<8| trans:0>{\tikz{\labeledcell{a,b}5}}
      \only<9->{\tikz{\labeledcell{a,b,\alert{c}}5}}
    \end{column}
  \end{columns}
\end{frame}
\end{document}
