\documentclass{beamer}
\usepackage{up}

\title{Функции}

\subtitle{(част 2)}

\date{23 ноември 2016 г.}

\usetikzlibrary{arrows.meta}

\begin{document}

\begin{frame}
  \titlepage
\end{frame}

\section{Програмен стек}

\begin{frame}
  \frametitle{Схема на програмната памет}

  \tikzset{
      m/.style={
      matrix of nodes,
      nodes={
        rectangle,
        minimum width=25ex,
        minimum height=2.5em
        }
    }
  }

  \begin{center}
    \begin{tikzpicture}
      \matrix (a) [m] {
        \ldots \\
        |[minimum height=5em,fill=diagramnavy]| Програмен стек \\
        |[fill=diagramnavy]| Статични данни \\
        |[fill=diagramblue]| Програмен код \\
        \ldots \\
      };
      \foreach\x in {east,west} { \draw (a-1-1.north \x) -- (a-5-1.south \x); }
      \foreach\x in {2,...,5} { \draw (a-\x-1.north west) -- (a-\x-1.north east); }
    \end{tikzpicture}
  \end{center}
\end{frame}

\begin{frame}
  \frametitle{Програмен стек}

  \tikzset{
      l/.style={-{Stealth[length=1.5ex]}},
      m/.style={
      matrix of nodes,
      nodes={
        rectangle,
        minimum width=25ex,
        minimum height=2em,
        fill=diagramblue,
        draw=black
      },
      nodes in empty cells
    }
  }

  \begin{center}
    \begin{tikzpicture}
        \matrix (a) [m] {
          |(sborder) [minimum height = 8em]| \ldots \\
          |(stop)|
          данни \\
          данни \\
          данни \\
          |(sbottom)|
          данни \\
        };
        \arrowto{sbottom.south west}{дъно на стека}{above left}
        \arrowto{stop.north east}{връх на стека}{below right}
        \arrowto{sborder.north east}{граница на стека}{below right}
    \end{tikzpicture}
  \end{center}
\end{frame}

\begin{frame}
  \frametitle{Програмен стек}

  \tikzset{
      l/.style={-{Stealth[length=1.5ex]}},
      m/.style={
      matrix of nodes,
      nodes={
        rectangle,
        minimum width=25ex,
        minimum height=2em,
        fill=diagramblue,
        draw=black
        }
    }
  }

  \begin{center}
    \begin{tikzpicture}
        \matrix (a) [m] {
          |(sborder) [minimum height = 6em]| \ldots \\
          |(stop)|
          данни \\
          данни \\
          данни \\
          данни \\
          |(sbottom)|
          данни \\
        };
        \arrowto{sbottom.south west}{дъно на стека}{above left}
        \arrowto{stop.north east}{връх на стека}{below right}
        \arrowto{sborder.north east}{граница на стека}{below right}
    \end{tikzpicture}
  \end{center}
\end{frame}

\begin{frame}
  \frametitle{Програмен стек}

  \tikzset{
      l/.style={-{Stealth[length=1.5ex]}},
      m/.style={
      matrix of nodes,
      nodes={
        rectangle,
        minimum width=25ex,
        minimum height=2em,
        fill=diagramblue,
        draw=black
        }
    }
  }

  % TODO: анимация на стека
  \begin{center}
    \begin{tikzpicture}
        \matrix (a) [m] {
          |(sborder) [minimum height = 12em]| \ldots \\
          |(stop)|
          данни \\
          |(sbottom)|
          данни \\
        };
        \arrowto{sbottom.south west}{дъно на стека}{above left}
        \arrowto{stop.north east}{връх на стека}{below right}
        \arrowto{sborder.north east}{граница на стека}{below right}
    \end{tikzpicture}
  \end{center}
\end{frame}


\begin{frame}
  \frametitle{Стекова рамка на функция}

  \tikzset{
      l/.style={-{Stealth[length=1.5ex]}},
      m/.style={
      matrix of nodes,
      nodes={
        rectangle,
        minimum width=25ex,
        minimum height=2em,
        fill=diagramblue,
        draw=black
        }
      },
      dots/.style={minimum height=5em}
  }

  \begin{center}
    \begin{tikzpicture}
      \matrix (a) [m] {
        |[dots]| \ldots \\
        локални променливи \\
        адрес за връщане \\
        фактически параметри \\
        |[dots]|  \ldots \\
      };
      \draw[very thick] (a-2-1.north west) rectangle (a-4-1.south east);
      \arrowto{a-2-1.south east}{рамков указател}{below right}
    \end{tikzpicture}
  \end{center}
\end{frame}

\begin{frame}
  \frametitle{Област за програмен код}

  \tikzset{
    l/.style={-{Stealth[length=1.5ex]}},
    m/.style={
      matrix of nodes,
      nodes={
        rectangle,
        minimum width=30ex,
        minimum height=2em,
        fill=diagramblue,
        draw=black
      }},
    dots/.style={minimum height=5em}
  }

  \begin{center}
    \begin{tikzpicture}
      \matrix (a) [m] {
        |[dots]| \ldots \\
        \tt{square} \\
        \tt{heron} \\
        |(t)| \tt{calculateTriangleArea} \\
        \tt{distance} \\
        \tt{main} \\
      };
      \arrowto{t.east}{програмен брояч}{below right}
    \end{tikzpicture}
  \end{center}
\end{frame}

\section{Предаване на параметри}

\begin{frame}
  \frametitle{Предаване по стойност (call by value)}

  \begin{itemize}[<+->]
  \item пресмята се стойността на фактическия параметър
  \item в стековата рамка на функцията се създава \alert{копие} на стойността
  \item всяка промяна на стойността остава локална за функцията
  \item при завършване на функцията, предадената стойност и всички промени над нея \alert{изчезват}
  \end{itemize}
\end{frame}

\begin{frame}
  \frametitle{Странични ефекти}

  Какви странични ефекти може да има функция на C++?\\[1em]
  \pause
  \begin{itemize}[<+->]
  \item Използване на глобални променливи
  \item Използване на статични променливи\\
    \tta{static} <дефиниция\_на\_променлива>
  \item Работа с вход или изход
  \end{itemize}
\end{frame}

\begin{frame}
  \frametitle{Предаване по псевдоним (call by reference)}

  \begin{itemize}[<+->]
  \item Понякога искаме промените във \textbf{формалните} параметри
    да се отразят във \textbf{фактическите} параметри
  \item Тогава трябва да обявим, че искаме фактическите параметри да могат да бъдат променяни
  \item{} <параметър> ::= <тип>\tta{\&} <идентификатор>
  \item \textbf{Примери:}
    \begin{itemize}
    \item \lst{int add5(int\& x) \{ x += 5; return x; \}}
    \item \alert{фактическият параметър трябва да е lvalue!}
    \item \sta{add5(3);}
    \item \lst{int a = 3; cout << add5(a) << ' ' << a;}
    \end{itemize}
  \end{itemize}
\end{frame}

\begin{frame}[fragile]
  \frametitle{Пример за предаване по псевдоним}

  Размяна на две променливи
\begin{lstlisting}
void swap(int& x, int& y) {
  int tmp = x;
  x = y;
  y = tmp;
}
\end{lstlisting}
\pause
\begin{lstlisting}
int main() {
  int a = 5, b = 8;
  swap(a, b);
  cout << a << ' ' << b << endl;
}
\end{lstlisting}
\end{frame}

\begin{frame}
  \frametitle{Стекова рамка при предаване по псевдоним}

  \tikzset{
    m/.style={
      matrix of nodes,
      nodes={
        text height=1ex,
        text depth=0.25ex,
        minimum height=1em,
        minimum width=8ex
      },
      nodes in empty cells,
      ampersand replacement=\&
    },
    dots/.style={minimum height=5em},
    table/.style={
      rectangle,
      very thin,
      minimum width=25ex,
      fill=diagramblue,
      draw=black
    }
  }

  \begin{center}
    \begin{tikzpicture}
      \matrix [m] {
       |(swap)| \tt{swap} \& \tt{tmp} \& |(tmp) [table]| \\
       \& \& |[table]| адрес на връщане \\
       \& \tt y \& |(y) [table]| \\
       \& \tt x \& |(x) [table]| \\
       |(main)| \tt{main} \& \tt b \& |(b) [table]|\\
       \& \tt a \& |(a) [table]| \\
       \& \& |(bottom) [table]| адрес на връщане \\
      };
      \draw[very thick] (swap.north west) -- (tmp.north west) rectangle (b.north east);
      \draw[very thick] (main.north west) -- (b.north west) rectangle (bottom.south east);
      \draw (y.east) to [bend left=45] (b.east);
      \draw (x.east) to [bend left=45] (a.east);
    \end{tikzpicture}
  \end{center}
\end{frame}

\begin{frame}
  \frametitle{Претоварване на функции (overloading)}

  \begin{itemize}[<+->]
  \item \textbf{Проблем:} Функцията \tt{swap} работи само за \tt{int}!
  \item Ако искаме функция за размяна на \tt{double} променливи, трябва да направим нова функция \lst{swap\_double(double\&, double\&);}
  \item Аналогично за \tt{char}, \tt{long}, \ldots
  \item Трябва ли да си измисляме нови имена за всяка от тези функции, които всъщност прави едно и също?
  \item \alert{Не!} Можем да използваме едно и също име за няколко функции!
  \item Сигнатурата на функцията зависи от:
    \begin{itemize}[<.->]
    \item типа на връщане
    \item типа и реда на параметрите
    \end{itemize}
  \item Функции с еднакво име и различна сигнатура са различни
  \item Казваме, че името е \alert{претоварено} (overloaded)
  \item \alert{Проблем}: може да възникне нееднозначност при извикването!
  \end{itemize}
\end{frame}

\begin{frame}
  \frametitle{Предаване на масиви като параметри}

  \small
  \begin{itemize}[<+->]
  \item{} <параметър\_масив> ::= <тип> <име> \tta[[<константен\_израз>]\tta]\\
    \hspace{26ex}<тип>\tta* <име>
  \item размерът на масива \alert{се игнорира}!
    \begin{itemize}
    \item затова обикновено се подава като допълнителен параметър
    \end{itemize}
  \item промените в масива винаги се отразяват в оригинала
  \item \textbf{Примери:}
    \begin{itemize}
    \item \lst{int readArray(int a[]);}
    \item \lst{void printArray(int* a, int n);}
    \end{itemize}
  \end{itemize}
\end{frame}

\begin{frame}
  \frametitle{Примерни функции}

  \begin{enumerate}
  \item Да се напише функция, която въвежда масив
  \item Да се напише функция, която извежда масив
  \item Да се напише функция, която търси елемент в масив
  \item Да се напише функция, която проверява дали два низа са равни
  \item Да се напише функция, която намира най-малкия и най-големия елемент на масив
  \end{enumerate}
\end{frame}

\begin{frame}
  \frametitle{Връщане на няколко резултата}

  \begin{itemize}[<+->]
  \item Как да върнем едновременно минимума и максимума?
  \item \textbf{Идея:} да използваме параметрите за резултат!
  \item Можем да запишем резултат в параметър, предаден по псевдоним
  \item \lst{void findMinMax(int a[], int n, int\& min, int\& max);}
  \end{itemize}
\end{frame}
\end{document}
