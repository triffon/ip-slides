\documentclass{beamer}
\usepackage{up}

\title{Функции от по-висок ред}

\usetikzlibrary{arrows.meta}

\date{11--18 януари 2017 г.}

\begin{document}

\begin{frame}
  \titlepage
\end{frame}

\section{Преговор: програмна памет}

\begin{frame}
  \frametitle{Схема на програмната памет}

  \tikzset{
      m/.style={
      matrix of nodes,
      nodes={
        rectangle,
        minimum width=25ex,
        minimum height=2.5em
        }
    }
  }

  \begin{center}
    \begin{tikzpicture}
      \matrix (a) [m] {
        \ldots \\
        |[minimum height=5em,fill=diagramnavy]| Програмен стек \\
        |[fill=diagramnavy]| Статични данни \\
        |[fill=diagramblue]| Програмен код \\
        \ldots \\
      };
      \foreach\x in {east,west} { \draw (a-1-1.north \x) -- (a-5-1.south \x); }
      \foreach\x in {2,...,5} { \draw (a-\x-1.north west) -- (a-\x-1.north east); }
    \end{tikzpicture}
  \end{center}
\end{frame}

\begin{frame}
  \frametitle{Област за програмен код}

  \tikzset{
    l/.style={-{Stealth[length=1.5ex]}},
    m/.style={
      matrix of nodes,
      nodes={
        rectangle,
        minimum width=30ex,
        minimum height=2em,
        fill=diagramblue,
        draw=black
      }},
    dots/.style={minimum height=5em}
  }

  \begin{center}
    \begin{tikzpicture}
      \matrix (a) [m] {
        |[dots]| \ldots \\
        \tt{square} \\
        \tt{heron} \\
        |(t)| \tt{calculateTriangleArea} \\
        \tt{distance} \\
        \tt{main} \\
      };
      \arrowto{t.east}{програмен брояч}{below right}
    \end{tikzpicture}
  \end{center}
\end{frame}

\section{Указатели към функции}

\begin{frame}
  \frametitle{Указател към функция}

  \begin{itemize}[<+->]
  \item Кодът на всяка функция на C++ се превежда до машинен код
  \item Машинният код на функциите е разположен в областта за програмен код
  \item \alert{Адресът на функцията} наричаме адресът на първата инструкция във функцията
  \item Можем да създаваме \alert{указатели към функции}
  \item Името на всяка функция може да се раглежда като константен указател към кода ѝ
  \item Стойността на указателя към функцията e адресът на нейния код
  \end{itemize}
\end{frame}

\begin{frame}
  \frametitle{Дефиниране на указатели към функции}

  {\small
    <тип> \tta{(*}<идентификатор>\tta)\tta(<формални параметри>\tta) [\tta= <указател> ]\tta;}
  \pause
  \begin{itemize}
  \item имената на параметрите могат да се пропуснат
  \end{itemize}
  \pause
  \textbf{Примери:}
  \begin{itemize}[<+->]
  \item \lst{void (*f)(int&, int&);}
  \item \lst{f = swap;}
  \item \lst{double (*op)(double) = sin;}
  \item \lst{op = cos;}
  \item \lst{op = NULL;}
  \item \sta{void (*p)(int, int\&) = f;}
  \item \sta{sin = op;}
  \end{itemize}
\end{frame}

\begin{frame}[fragile]
  \frametitle{Извикване на функция през указател}

\begin{lstlisting}
void (*f)(int&, int&) = swap;
int x = 5, y = 8;
\end{lstlisting}
  \pause
  Три еквивалентни начина за извикване на функията:
  \begin{itemize}
  \item \lst{swap(x, y);}
  \item \lst{(*f)(x, y);}
  \item \lst{f(x, y);}
  \end{itemize}
\end{frame}

\section{Потребителски типове}

\begin{frame}
  \frametitle{Дефиниране на потребителски типове}

  В C++ е позволено дефиниране на потребителски типове:\\
  \alt<+>{
    \tta{typedef} <декларация\_на\_променлива>\tta;}{
    \tta{typedef} <тип> <име>\tta;}
  \onslide<+->
  \begin{itemize}
  \item създава се нов тип <име>, който е еквивалентен на <тип>
  \end{itemize}
  \onslide<+->
  \textbf{Примери:}
  \begin{itemize}[<+->]
  \item \lst{typedef int number;}
  \item \lst{number x;} \eqv \lst{int x;}
  \item \lst{number f(number y) \{ ... \}} \eqv \lst{int f(int y) \{ ... \}}
  \item \lst{typedef double matrix[5][10];}
  \item \lst{matrix a;} \eqv \lst{double a[5][10];}
  \item \lst{typedef int** pointer2;}
  \item \lst{int x; int* p = &x; pointer2 q = &p;}
  \item \lst{typedef int& ref;}
  \item \lst{ref y = x;}
  \end{itemize}
\end{frame}

\begin{frame}
  \frametitle{Потребителски типове за указатели към функции}

  \begin{itemize}[<+->]
  \item \lst{typedef double (*mathfun)(double);}
  \item \lst{mathfun p = exp; p = log;}
  \item \lst{typedef void (*procedure)();}
  \item \lst{void h() \{ cout << "h()\\n"; \}}
  \item \lst{procedure q = h; q();}
  \item \lst{void (*r)() = q;}
  \end{itemize}
\end{frame}

\section{Функции от по-висок ред}

\begin{frame}[fragile]
  \frametitle{Примерна сума 1}

  \textbf{Задача 1.}\\
  Да се пресметне сумата $\sin(1) + \sin(2) + \sin(3) + \ldots + \sin(n)$.\\[1em]
  \pause
  \textbf{Решение:}
\begin{lstlisting}
double sum_sin(int n) {
  double s = 0;
  for(int i = 1; i <= n; i++)
    s += sin(i);
  return s;
}
\end{lstlisting}
\end{frame}

\begin{frame}[fragile]
  \frametitle{Примерна сума 2}

  \textbf{Задача 2.}\\
  Да се пресметне сумата $\cos(1) + \cos(2) + \cos(4) + \ldots + \cos(n)$.\\[1em]
  \pause
  \textbf{Решение:}
\begin{lstlisting}
double sum_cos(int n) {
  double s = 0;
  for(int i = 1; i <= n; i *= 2)
    s += cos(i);
  return s;
}
\end{lstlisting}
\end{frame}

\begin{frame}[fragile]
  \frametitle{Открийте разликите!}

\begin{lstlisting}
double @\alert<2>{sum\_sin}@(int n) {
  double s = 0;
  for(int i = 1; i <= n; i@\alert<2>{++}@)
    s += @\alert<2>{sin}@(i);
  return s;
}

double @\alert<2>{sum\_cos}@(int n) {
  double s = 0;
  for(int i = 1; i <= n; i @\alert<2>{*= 2}@)
    s += @\alert<2>{cos}@(i);
  return s;
}
\end{lstlisting}
\end{frame}

\begin{frame}[fragile]
  \frametitle{Общият шаблон}

\begin{lstlisting}
double @\alert{<name>}@(int n) {
  double s = 0;
  for(int i = 1; i <= n; i = @\alert{<next>}@(i)})
    s += @\alert{<f>}@(i);
  return s;
}
\end{lstlisting}
\end{frame}

\begin{frame}[fragile]
  \frametitle{Функциите като параметри}

\begin{lstlisting}
double sum(int n, double (*f)(double), int (*next)(int)) {
  double s = 0;
  for(int i = 1; i <= n; i = next(i))
    s += f(i);
  return s;
}
\end{lstlisting}
  \pause
  \begin{itemize}[<+->]
  \item \lst!int plus1(int i) { return i + 1; }!
  \item \lst{sum_sin(n)} \eqv \lst{sum(n, sin, plus1)}
  \item \lst!int mult2(int i) { return i * 2; }!
  \item \lst{sum_cos(n)} \eqv \lst{sum(n, cos, mult2)}
  \end{itemize}
\end{frame}

\begin{frame}[fragile]
  \frametitle{Произведение от по-висок ред}

\begin{lstlisting}
double product(int n, double (*f)(double), int (*next)(int)) {
  double s = 1;
  for(int i = 1; i <= n; i = next(i))
    s *= f(i);
  return s;
}
\end{lstlisting}
  \pause
  \textbf{Примери:}
  \begin{itemize}[<+->]
  \item \textbf{Задача.} Да се пресметне произведението $\tan(1)\tan(2)\tan(3)\ldots\tan(n)$.
  \item \textbf{Решение.} \lst{product(n, tan, plus1);}
  \end{itemize}
\end{frame}

\begin{frame}[fragile]
  \frametitle{Откийте разликите 2.0}

\begin{lstlisting}
double @\alert<2>{sum}@(int n, double (*f)(double), int (*next)(int)) {
  double s = @\alert<2>0@;
  for(int i = 1; i <= n; i = next(i))
    s @\alert<2>+@= f(i);
  return s;
}

double @\alert<2>{product}@(int n, double (*f)(double), int (*next)(int)) {
  double s = @\alert<2>1@;
  for(int i = 1; i <= n; i = next(i))
    s @\alert<2>*@= f(i);
  return s;
}
\end{lstlisting}
\end{frame}

\begin{frame}[fragile]
  \frametitle{Натрупване от по-висок ред (accumulate)}
    Да се напише функция, която пресмята натрупването
  \begin{equation*}
    \bot \oplus f(a) \oplus f(next(a)) \oplus f(next(next(a))) \oplus \ldots \oplus f(b)
  \end{equation*}
  където $\oplus$ е двуместна операция,\\
  а $\bot$ е нейната ``нулева стойност'', т.е. $x\oplus\bot = x$.\\[1em]
  \pause
  \textbf{Решение:}
  \begin{itemize}
  \item \lst{typedef int (*nextfun)(int);}
  \item \lst{typedef double (*mathfun)(double);}
  \item \lst{typedef double (*mathop)(double, double);}
  \end{itemize}
  \pause

\begin{lstlisting}
double accumulate (mathop op, double base_value,
                   double a,  double b,
                   mathfun f, nextfun  next);
\end{lstlisting}
\end{frame}

\begin{frame}[fragile]
  \frametitle{Натрупване от по-висок ред (accumulate)}
\begin{lstlisting}
double accumulate (mathop op, double base_value,
                   double a,  double b,
                   mathfun f, nextfun next) {
  double s = base_value;
  for(int i = a; i <= b; i = next(i))
    s = op(s, f(i));
  return s;
}
\end{lstlisting}
  \pause
  \textbf{Примери:}
  \small
  \begin{itemize}[<+->]
  \item \lst!double plus(double a, double b) { return a + b; }!
  \item \lst{sum(n, f, next)} \eqv \lst{accumulate(plus, 0, 1, n, f, next)}
  \item \lst!double mult(double a, double b) { return a * b; }!
  \item \lst{product(n, f, next)} \eqv \lst{accumulate(mult, 1, 1, n, f, next)}
  \end{itemize}
\end{frame}

\begin{frame}
  \frametitle{Задачи за accumulate}
  С помощта на \tt{accumulate} да се пресметнат:

  \begin{enumerate}[<+->]
  \item $n!$
  \item $x^n$
  \item $\displaystyle\sum_{i=0}^n \frac{x^i}{i!}$
  \item $\displaystyle\left(\begin{array}{@{}c@{}}n\\k\end{array}\right)$
  \item $\displaystyle x - \frac{x^3}{3!} + \frac{x^5}{5!} - \frac{x^7}{7!} + \ldots + (-1)^n\frac{x^{2n+1}}{(2n+1)!}$
  \end{enumerate}
\end{frame}

\begin{frame}[fragile]
  \frametitle{Анонимни функции в C++14}
  \begin{itemize}[<+->]
  \item \tta{[](}<параметри>\tta{) -> }<тип> \tta{\{}<тяло>\tta{\}}
  \item създава анонимна ($\lambda$) функция, дефинирана като:\\
    <тип> $\lambda$\tt(<параметри>\tt{) \{}<тяло>\tt\}
  \item ако <тяло> е от вида \lst{return }<израз>\tt; можем да пропуснем <тип>:
  \item<.-> \tta{[](}<параметри>\tta{) \{}<тяло>\tta{\}}
  \item \textbf{Примери:}
  \item \lst!sum(n, sin, [](int n) { return n + 1; })!
  \item \lst!accumulate([](double a, double b) { return a * b;}, 1,!\\
    \hspace{13ex}\lst!1, n, [](double x) {return x;},!\\
    \hspace{13ex}\lst![](int n) { return n + 1;})!
  \end{itemize}
\end{frame}

\section{Функциите като върнат резултат}

\begin{frame}[fragile]
  \frametitle{Функциите като върнат резултат}

  \textbf{Задача.} Да се напише функция, която по зададен числов код 1, 2, 3 или 4, връща съответно една от функциите $\sin$, $\cos$, $e^x$, $\log$.\\[1em]
  \pause\onslide<+->
  \textbf{Решение:}
  \temporal<+>{\lst{? choose_function(int n)}}{\lst{double (*choose_function(double))(int n);}}{\lst{mathfun choose_function(int n);}}
  \onslide<+(1)->
\begin{lstlisting}
mathfun choose_function(int n) {
  switch(n) {
    case 1 : return sin;
    case 2 : return cos;
    case 3 : return exp;
    case 4 : return log;
    default : return NULL;
}
\end{lstlisting}
\end{frame}

\begin{frame}[fragile]
  \frametitle{Производна}

  \textbf{Задача.}
  Да се напише функция, която по дадена едноаргументна функция $f$ връща нейната производна.\\[1em]
  \pause
  \textbf{Преговор:}
  \begin{equation*}
    f'(x) \alt<+>{= \lim_{\Delta x\to 0} \frac{f(x+\Delta x) - f(x)}{\Delta x}}{\approx\frac{f(x+\Delta x) - f(x)}{\Delta x}\;\text{ за малки }\Delta x}
  \end{equation*}
  \pause
  \textbf{Решение.}
\begin{lstlisting}
  matfun derive(matfun f) {
    return ?
  }
\end{lstlisting}
  \pause
  \alert{Проблем:} За различни \tt{f}, трябва да връщаме различна функция, но предварително не знаем коя!
\end{frame}

\begin{frame}[fragile]
  \frametitle{Намиране на производна}

  \alert{Идея №1.} Ще използваме помощна функция.\pause
\begin{lstlisting}
const double EPS = 1E-10;
double derivative(double x) {
  return (function(x + EPS) - function(x)) / EPS;
}
\end{lstlisting}
  \pause
  \alert{Проблем №2:} Как да подадем \tt{function}?\\
  \pause
  \alert{Идея №3}: Да използваме глобална променлива.\\
  \pause
\begin{lstlisting}
matfun function = NULL;
matfun derive(mathfun f) {
  function = f;
  return derivative;
}
\end{lstlisting}
  \pause
  \alert{Проблем №3.} Грозно е.\\
  \pause
  \alert{Проблем №4.} Може да работи само с една производна в даден момент.
\end{frame}

\begin{frame}[fragile]
  \frametitle{Намиране на производна с анонимни функции}
  \alert{Идея №4:} Да използваме анонимни функции!
  \pause
\begin{lstlisting}
auto derive(mathfun f) {
  return [f](double x) { return (f(x + EPS) - f(x)) / EPS; };
}
\end{lstlisting}
  \pause
  \begin{itemize}[<+->]
  \item \tt{[f]} означава, че позволяваме на анонимната функция да използва копие на указателя \tt{f}.
  \item\lst{auto} означава, че искаме C++14 сам да се сети за типа на връщания резултат (\alert{\tt{mathfun} вече не върши работа})\\
  \end{itemize}
  \onslide<+->
  \textbf{Примери:}
  \begin{itemize}[<+->]
  \item \lst{auto mycos = derive(sin);}
  \item \lst{cout << mycos(0) << ' ' << cos(0);}
  \item \lst{cout << exp(1) << ' ' << derive(exp)(1);}
  \end{itemize}
\end{frame}
\end{document}
