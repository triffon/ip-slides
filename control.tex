\documentclass{beamer}
\usepackage{up}

\title{Управляващи оператори в C++}

\date{19 октомври 2016 г.}

\begin{document}

\begin{frame}
  \titlepage
\end{frame}

\section{Изчислителни процеси}

\begin{frame}
  \frametitle{Изчислителни процеси}

  \begin{itemize}
  \item Алгоритъм: последователност от стъпки за извършване на пресмятане
  \item Блок-схема
  \end{itemize}\ \\[1em]
  \begin{tikzpicture}
    [node distance=4ex,font=\small]
    \node (begin)  [entry]                  {начало};
    \node (input)  [io,right=of begin]      {%
      \begin{tabular}{l}
        въведи \tt a\\
        въведи \tt b        
      \end{tabular}
    };
    \node (assign) [op,right=of input]      {\tt{x = -b/a}};
    \node (output) [io,right=of assign]     {изведи \tt x};
    \node (end)    [entry,right=of output]  {край};

    \graph[use existing nodes,edges=thick]
    { begin -> input -> assign -> output -> end };
  \end{tikzpicture}\\[2em]
  Пример за линеен процес
\end{frame}

\begin{frame}
  \frametitle{Разклоняващи се процеси}

  \begin{tikzpicture}
    [node distance=4ex,font=\scriptsize]
    \node (begin)  [entry]                  {начало};
    \node (input)  [io,right=of begin]      {%
      \begin{tabular}{l}
        въведи \tt a\\
        въведи \tt b
      \end{tabular}
    };
    \node (a0)     [cond,right=of input]    {\tt{a == 0}};
    \node (assign) [op,right=of a0]         {\tt{x = -b/a}};
    \node (output) [io,right=of assign]     {изведи \tt x};
    \node (p1)     [point,right=of output]  {};
    \node (b0)     [cond,below=of a0]       {\tt{b == 0}};
    \node (nosol)  [io,right=of b0]         {
      \begin{tabular}{l}
        изведи\\
        \tt{"няма решение"}
      \end{tabular}};
    \node (p2)     [point] at (nosol -| p1) {};
    \node (allsol) [io,below=of b0]         {
      \begin{tabular}{l}
        изведи\\
        \tt{"всяко x е решение"}
      \end{tabular}};
    \node (end)    [entry] at (allsol -| p1){край};    

    \graph[use existing nodes,edges=thick]
    { begin -> input -> a0 ->[edge label={не}] assign -> output -- p1 -- p2 -> end;
      a0 ->[edge label=да] b0 ->[edge label={не}] nosol -> p2;
      b0 ->[edge label=да] allsol -> end;
    };
  \end{tikzpicture}
\end{frame}

\begin{frame}
  \frametitle{Индуктивни циклични процеси}
  
  \begin{tikzpicture}
    [node distance=4ex,font=\scriptsize]
    \node (begin)  [entry]                  {начало};
    \node (input)  [io,right=of begin]      {въведи \tt n};
    \node (init)   [op,right=of input]      {
      \begin{tabular}{l}
        \tt{i = 1}\\
        \tt{f = 1}
      \end{tabular}};
    \node (p0)     [point,below=of init]    {};
    \node (guard)  [cond,below=of p0]       {\tt{i <= n}};
    \node (body)   [op,below=of guard]      {
      \begin{tabular}{l}
        \tt{f = f * i}\\
        \tt{i = i + 1}
      \end{tabular}};
    \node (output) [io,right=of guard]      {изведи \tt f};
    \node (end)    [entry,right=of output]  {край};

    \graph[use existing nodes,edges=thick]
    { begin -> input -> init -- p0 -> guard ->[edge label={не}] output -> end;
      guard ->[edge label=да] body ->[to path={-- ++(-10ex,0) |- (p0)}] p0; };
  \end{tikzpicture}\\[-10em]
  \pause
  \begin{tabular}{|c|c|c|}
    \hline
    \tt n&\tt i&\tt f\\
    \hline\hline
    4&1&1\\
    \hline\tpause
    4&2&1\\
    \hline\tpause
    4&3&2\\
    \hline\tpause
    4&4&6\\
    \hline\tpause
    4&5&24\\
    \hline
  \end{tabular}
\end{frame}

\begin{frame}
  \frametitle{Итеративни циклични процеси}

  \begin{tikzpicture}
    [node distance=4ex,font=\scriptsize]
    \node (begin)  [entry]                  {начало};
    \node (input)  [io,right=of begin]      {
      \begin{tabular}{l}
        въведи \tt a\\
        въведи \tt b
      \end{tabular}};
    \node (guard)   [cond,right=of input,inner sep=-2pt]   {
      \begin{tabular}{c}
        \tt{a == 0}\\
        или\\
        \tt{b == 0}
      \end{tabular}};
    \node (p)       [point,below=of guard] {};
    \node (nogcd)   [io,right=of guard] {
      \begin{tabular}{l}
        изведи\\
        \tt{"грешка"}
      \end{tabular}};
    \node (end)     [entry,right=of nogcd] {край};
    \node (ab)      [cond,below=of p]  {\tt{a == b}};
    \node (ab1)     [cond,below=of ab]     {\tt{a > b}};
    \node (output)  [io] at (ab -| end)    {изведи \tt a};
    \node (a)       [op,left=of ab1]       {\tt{a = a - b}};
    \node (b)       [op,right=of ab1]       {\tt{b = b - a}};

    \graph[use existing nodes,edges=thick]
    { begin -> input -> guard ->[edge label=да] nogcd -> end;
      guard --[edge label=не] p -> ab ->[edge label=да] output -> end;
      ab ->[edge label=не] ab1 ->[edge label=да] a ->[to path={|- (p)}] p;
      ab1 ->[edge label=не] b ->[to path={|- (p)}] p;};
  \end{tikzpicture}\\[-10em]
  \pause
  \begin{tabular}{|c|c|}
    \hline
    \tt a&\tt b\\
    \hline\hline
    12&32\\
    \hline\tpause
    12&20\\
    \hline\tpause
    12&8\\
    \hline\tpause
    4&8\\
    \hline\tpause
    4&4\\
    \hline
  \end{tabular}
\end{frame}

\begin{frame}
  \frametitle{Структурни езици — разклонение}

  \begin{columns}
    \begin{column}{.5\textwidth}
      \begin{enumerate}
      \item Въведи \tt a, \tt b
      \item Ако \tt{a == 0}, към 5
      \item \tt{x = -b / a}
      \item Премини към 9
      \item Ако \tt{b = 0}, към 8
      \item ``Няма решения''
      \item Премини към 9
      \item ``Всяко $x$ е решение''
      \item Край
      \end{enumerate}
    \end{column}
    \begin{column}{.5\textwidth}
      \begin{itemize}
      \item Въведи \tt a, \tt b
      \item Ако \tt{a == 0}
        \begin{itemize}
        \item Ако \tt{b == 0}
          \begin{itemize}
          \item ``Всяко $x$ е решение''
          \end{itemize}
        \item Иначе
          \begin{itemize}
          \item ``Няма решения''
          \end{itemize}
        \end{itemize}
      \item Иначе
        \begin{itemize}
        \item \tt{x = -b / a}
        \end{itemize}
      \end{itemize}
    \end{column}
  \end{columns}
\end{frame}

\begin{frame}
  \frametitle{Структурни езици — индуктивен цикъл}

  \begin{columns}
    \begin{column}{.5\textwidth}
      \begin{enumerate}
      \item Въведи \tt n
      \item \tt{i = 1}
      \item \tt{f = 1}
      \item Ако \tt{i > n}, към 8
      \item \tt{f = f * i}
      \item \tt{i = i + 1}
      \item Премини към 4
      \item Изведи \tt f
      \item Край
      \end{enumerate}
    \end{column}
    \begin{column}{.5\textwidth}
      \begin{itemize}
      \item Въведи \tt n
      \item \tt{i = 1}
      \item \tt{f = 1}
      \item Повтаряй \tt n пъти
        \begin{itemize}
        \item \tt{f = f * i}
        \item \tt{i = i + 1}
        \end{itemize}
      \item Изведи \tt f
      \end{itemize}
    \end{column}
  \end{columns}
\end{frame}

\begin{frame}
  \frametitle{Структурни езици — итеративен цикъл}

  \begin{columns}[t,onlytextwidth]
    \begin{column}{.5\textwidth}
      \begin{enumerate}
      \item Въведи \tt a, \tt b
      \item Ако \tt{a == b}, към 6.
      \item Ако \tt{a > b}, към 5.
      \item \tt{b = b - a}; към 2.
      \item \tt{a = a - b}; към 2.
      \item Изведи \tt a
      \item Край
      \end{enumerate}
    \end{column}
    \begin{column}{.5\textwidth}
      \begin{itemize}
      \item Въведи \tt a, \tt b
      \item Докато \tt{a != b}
        \begin{itemize}
        \item Ако \tt{a > b}
          \begin{itemize}
          \item \tt{a = a - b}
          \end{itemize}
        \item В противен случай
          \begin{itemize}
          \item \tt{b = b - a}
          \end{itemize}
        \end{itemize}
      \item Изведи \tt a
      \end{itemize}
    \end{column}
  \end{columns}
\end{frame}

\section{Основни понятия}

\begin{frame}
  \frametitle{Основни понятия}

  \begin{itemize}
  \item \textbf{Операция} (operator)
  \item \textbf{Израз} (expression)
  \item \textbf{Оператор/команда} (statement)
  \item{}
    \begin{tabular}[t]{@{}rl@{}}
      <израз> ::= &<константа> | <променлива> |\\
      &<едноместна\_опeрация> <израз> |\\
      &<израз> <двуместна\_операция> <израз>
    \end{tabular}
  \item{} <оператор> ::= <израз>\tta;
  \end{itemize}
\end{frame}

\begin{frame}
  \frametitle{Оператор за присвояване}

  \begin{itemize}
  \item{} <променлива> \tta= <израз>\tta;
  \item{} <lvalue> \tta= <rvalue>\tta;
  \item{} <lvalue> --- място в паметта със стойност, която може да се променя
  \item Пример: променлива
  \item{} <rvalue> --- временна стойност, без специално място в паметта
  \item Пример: константа, литерал, резултат от пресмятане
  \item стандартно преобразуване на типовете:\\
    <rvalue> се преобразува до типа на <lvalue>
  \end{itemize}

\end{frame}

\begin{frame}
  \frametitle{Присвояването като операция}
  \begin{itemize}
  \item<1-> \alert{дясноасоциативна} операция
  \item<2-> \temporal<3>{\tt{a = b = c = 2;}}{\tt{a = (b = (c = 2));}}{$\underbrace{\tt{a = }\underbrace{\tt{(b = }\underbrace{\tt{(c = 2)}}_{\tt c})}_{\tt b}}_{\tt a}$}
  \item<3> \sta{(((a = b) = c) = 2);}
  \item<5-> Пример: \lstinline{cout << x + (b = 2);}
  \item<6-> Пример: \lstinline{(a = b) = a + 3;}
  \end{itemize}
\end{frame}

\begin{frame}
  \frametitle{Операция за изброяване}

  \begin{itemize}
  \item{} <израз1>\tta, <израз2>
  \item оценява и двата израза, но крайният резултат е оценката на втория израз
  \item \tt{a, b, c, d} $\Leftrightarrow$ \tt{(a, (b, (c, d)))}
  \item \alert{дясноасоциативна}
  \item използва се рядко
  \item Пример: \lstinline{a = (cout << x, x);}
  \end{itemize}
\end{frame}

\begin{frame}
  \frametitle{Съкратени оператори за присвояване}

  \begin{itemize}[<+->]
  \item \tt{a = a + 2} $\Leftrightarrow$ \tt{a += 2}
  \item \tt{-=}, \tt{*=}, \tt{/=}, \tt{\%=}
  \item \tt{a = a + 1}  $\Leftrightarrow$ \tt{++a}
  \item \tt{a = a - 1}  $\Leftrightarrow$ \tt{-{}-a}
  \item \tt{a++} увеличава a с 1, но връща предишната стойност на \tt a
    \begin{itemize}
    \item \tt{a++} $\Leftrightarrow$ \tt{(a = (tmp = a) + 1, tmp)}
    \end{itemize}
  \item \tt{a-{}-} действа аналогично
  \item \tta{a++} връща a, което е \alert{<lvalue>}
    \begin{itemize}
    \item Пример: \lst{++a += 5;}
    \end{itemize}
  \item \tta{a-{}-} връща предишната стойност на a, което е \alert{<rvalue>}
    \begin{itemize}
    \item Пример: \lst{x = a++ * b;} \sta{a++ += 5;}
    \end{itemize}
  \end{itemize}
\end{frame}

\begin{frame}[fragile]
  \frametitle{Оператор за блок}

  \begin{itemize}[<+->]
  \item \tta\{ \{ <оператор> \} \tta\}
  \item \tta\{ <оператор$_1$> <оператор$_2$> \ldots <оператор$_n$> \tta\}
  \item Вложени блокове\\
\begin{lstlisting}
{
  int x = 2;
  {
    x += 2;
    cout << x;
  }
}
\end{lstlisting}
  \end{itemize}
\end{frame}

\begin{frame}<0>
  \frametitle{Област на действие (scope)}
  
  \begin{itemize}[<+->]
  \item областта на действие се простира от дефиницията на
    променливата до края на блока, в който е дефинирана
  \item дефиниция на променлива със същото име в същия блок е
    забранена
  \item дефиниция на променлива във вложен блок покрива всички външни
    дефиниции със същото име
  \end{itemize}
  %TBD: пример
\end{frame}

\begin{frame}
  \frametitle{Празен оператор}
  \begin{itemize}
  \item \tta;
  \item \tt; $\Leftrightarrow$ \tt{\{\}}
  \item няма никакъв ефект
  \end{itemize}
\end{frame}

\begin{frame}
  \frametitle{Условен оператор}
  
  \begin{itemize}[<+->]
  \item \tta{if (}<израз>\tta) <оператор> [\tta{else} <оператор>]
  \item Съкратената форма $\Leftrightarrow$ пълна форма с празен оператор
    \begin{itemize}
    \item \lst{if (A) X;} $\Leftrightarrow$ \lst{if (A) X; else;}
    \end{itemize}
  \item Пример: \lst{if ( x < 2 ) y = 2;}
  \item Пример: \lst{if ( x > 5 ) y = 5; else y = 3;}
  \item Представяне на логически операции с вложени \text{if}:
    \begin{itemize}
    \item \lst{if (!A) X; else Y;} $\Leftrightarrow$ \lst{if (A) Y; else X;}
    \item \lst{if (A && B) X; else Y;} $\Leftrightarrow$ \lst{if (A) if (B) X; else Y; else Y;}
    \item \lst{if (A || B) X; else Y;} $\Leftrightarrow$ \lst{if (A) X; else if (B) X; else Y;}
    \end{itemize}
  \end{itemize}
\end{frame}
\end{document}

\begin{frame}
  \frametitle{Условна \alert{операция}}
  \begin{itemize}[<+->]
  \item{} <булев\_израз> ? <израз> : <израз>
  \item триместна (тернарна) операция
  \item Пример: \lst{x = (y < 2) ? y + 1 : y - 2;}
  \item \lst{A} $\Leftrightarrow$ \lst{A ? true : false}
  \item \lst{!A} $\Leftrightarrow$ \lst{A ? false : true}
  \item \lst{A && B} $\Leftrightarrow$ \lst{A ? B : false}
  \item \lst{A || B} $\Leftrightarrow$ \lst{A ? true : B}
  \end{itemize}
\end{frame}

\begin{frame}
  \frametitle{Задачи за условен оператор}
  
  \begin{enumerate}[<+->]
  \item Да се провери дали три числа образуват растяща редица
  \item Да се намери най-малкото от три числа
  \item Да се подредят три числа в растяща редица
  \item Да се провери дали три числа образуват Питагорова тройка
  \end{enumerate}
\end{frame}

\begin{frame}[fragile]
  \frametitle{Оператор за многозначен избор}
  
  \begin{itemize}
  \item
    \tta{switch (}<израз>\tta{) \{}\\
    \hspace{3ex}\{ \tta{case }<константен\_израз> \tta: \{ <оператор> \} \}\\
    \hspace{3ex}[ \tta{default :} \{ <оператор> \} ]\\
    \tta\}
  \item Пример:
\begin{lstlisting}
switch (x) {
	case 1 : x++;
	case 2 : x += 2;
	default : x += 5;
}
\end{lstlisting}
  \end{itemize}
\end{frame}

\begin{frame}[fragile]
  \frametitle{Оператор за прекъсване}
  
  \begin{itemize}
  \item \tta{break;}
  \item Пример:
\begin{lstlisting}
switch (x) {
	case 1 : x++; break;
	case 2 : x += 2; break;
	default : x += 5;
}
\end{lstlisting}
  \end{itemize}
\end{frame}

\begin{frame}
  \frametitle{Задачи за многозначен избор}
  
  \begin{enumerate}[<+->]
  \item Да се пресметне избрана от потребителя целочислена аритметична операция
  \item Да се провери дали дадена буква е гласна или съгласна
  \end{enumerate}
\end{frame}
\end{document}
