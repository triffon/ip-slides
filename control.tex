\documentclass{beamer}
\usepackage{up}

\title{Управляващи оператори C++}

\date{18 октомври 2016 г.}

\begin{document}

\begin{frame}
  \titlepage
\end{frame}

\section{Изчислителни процеси}

\begin{frame}
  \frametitle{Изчислителни процеси}

  \begin{itemize}
  \item Алгоритъм: последователност от стъпки за извършване на пресмятане
  \item Блок-схема
  \end{itemize}
  %TBD
  Пример за линеен процес
\end{frame}

\begin{frame}
  \frametitle{Разклоняващи се процеси}

  %TBD
\end{frame}

\begin{frame}
  \frametitle{Индуктивни циклични процеси}

  %TBD
  \begin{tabular}{|c|c|c|}
    \hline
    \tt n&\tt i&\tt f\\
    \hline\hline
    4&1&1\\
    \hline
    4&2&1\\
    \hline
    4&3&2\\
    \hline
    4&4&6\\
    \hline
    4&5&24\\
    \hline
  \end{tabular}
\end{frame}

\begin{frame}
  \frametitle{Итеративни циклични процеси}

  %TBD
  \begin{tabular}{|c|c|}
    \hline
    \tt a&\tt b\\
    \hline\hline
    12&32\\
    \hline
    12&20\\
    \hline
    12&8\\
    \hline
    4&8\\
    \hline
    4&4\\
    \hline
  \end{tabular}
\end{frame}

\begin{frame}
  \frametitle{Структурни езици — разклонение}

  \begin{columns}
    \begin{column}{.5\textwidth}
      \begin{enumerate}
      \item Въведи \tt a, \tt b
      \item Ако \tt{a == 0}, към 5
      \item \tt{x = -b / a}
      \item Премини към 9
      \item Ако \tt{b = 0}, към 8
      \item ``Няма решения''
      \item Премини към 9
      \item ``Всяко $x$ е решение''
      \item Край
      \end{enumerate}
    \end{column}
    \begin{column}{.5\textwidth}
      \begin{itemize}
      \item Въведи \tt a, \tt b
      \item Ако \tt{a == 0}
        \begin{itemize}
        \item Ако \tt{b == 0}
          \begin{itemize}
          \item ``Всяко $x$ е решение''
          \end{itemize}
        \item Иначе
          \begin{itemize}
          \item ``Няма решения''
          \end{itemize}
        \end{itemize}
      \item Иначе
        \begin{itemize}
        \item \tt{x = -b / a}
        \end{itemize}
      \end{itemize}
    \end{column}
  \end{columns}
\end{frame}

\begin{frame}
  \frametitle{Структурни езици — итеративен цикъл}

  \begin{columns}
    \begin{column}{.5\textwidth}
      \begin{enumerate}
      \item Въведи \tt n
      \item \tt{i = 1}
      \item \tt{f = 1}
      \item Ако \tt{i > n}, към 8
      \item \tt{f = f * i}
      \item \tt{i = i + 1}
      \item Премини към 4
      \item Изведи \tt f
      \item Край
      \end{enumerate}
    \end{column}
    \begin{column}{.5\textwidth}
      \begin{itemize}
      \item Въведи \tt n
      \item \tt{i = 1}
      \item \tt{f = 1}
      \item Повтаряй \tt n пъти
        \begin{itemize}
        \item \tt{f = f * i}
        \item \tt{i = i + 1}
        \end{itemize}
      \item Изведи \tt f
      \end{itemize}
    \end{column}
  \end{columns}
\end{frame}

\begin{frame}
  \frametitle{Структурни езици — индуктивен цикъл}

  \begin{columns}[t,onlytextwidth]
    \begin{column}{.5\textwidth}
      \begin{enumerate}
      \item Въведи \tt a, \tt b
      \item Ако \tt{a == b}, към 6.
      \item Ако \tt{a > b}, към 5.
      \item \tt{b = b - a}; към 2.
      \item \tt{a = a - b}; към 2.
      \item Изведи \tt a
      \item Край
      \end{enumerate}
    \end{column}
    \begin{column}{.5\textwidth}
      \begin{itemize}
      \item Въведи \tt a, \tt b
      \item Докато \tt{a != b}
        \begin{itemize}
        \item Ако \tt{a > b}
          \begin{itemize}
          \item \tt{a = a - b}
          \end{itemize}
        \item В противен случай
          \begin{itemize}
          \item \tt{b = b - a}
          \end{itemize}
        \end{itemize}
      \item Изведи \tt a
      \end{itemize}
    \end{column}
  \end{columns}
\end{frame}

\section{Основни понятия}

\begin{frame}
  \frametitle{Основни понятия}

  \begin{itemize}
  \item Операция (operator)
  \item Израз (expression)
  \item Оператор/команда (statement)
  \item{}
    \begin{tabular}[t]{@{}rl@{}}
      <израз> ::= &<константа> | <променлива> |\\
      &<едноместна\_опeрация> <израз> |\\
      &<израз> <двуместна\_операция> <израз>
    \end{tabular}
  \item{} <оператор> ::= <израз>\tta;
  \end{itemize}
\end{frame}

\begin{frame}
  \frametitle{Оператор за присвояване}

  \begin{itemize}
  \item{} <променлива> \tta= <израз>\tta;
  \item{} <lvalue> \tta= <rvalue>\tta;
  \item{} <lvalue> означава място в паметта със стойност, която може да
    се променя
  \item Пример: променлива
  \item{} <rvalue> означава временна стойност, без специално място в
    паметта
  \item стандартно преобразуване на типовете
  \end{itemize}

\end{frame}

\begin{frame}[<1-3>]
  \frametitle{Присвояването като операция}
  \begin{itemize}
  \item<1-> \alert{дясноасоциативна} операция
  \item<2-> \temporal<3>{\tt{a = b = c = 2;}}{\tt{a = (b = (c = 2));}}{$\underbrace{\tt{a = }\underbrace{\tt{(b = }\underbrace{\tt{(c = 2)}}_{\tt c})}_{\tt b}}_{\tt a}$}
  \item<3> \sta{(((a = b) = c) = 2);}
  \item<5-> Пример: \lstinline{cout << x + (b = 2);}
  \item<6-> Пример: \lstinline{(a = b) = a + 3;}
  \end{itemize}
\end{frame}

\begin{frame}
  \frametitle{Операция за изброяване}

  \begin{itemize}
  \item{} <израз1>\tta, <израз2>
  \item оценява и двата израза, но крайният резултат е оценката на втория израз
  \item \tt{a, b, c, d} $\Leftrightarrow$ \tt{(a, (b, (c, d)))}
  \item \alert{дясноасоциативна}
  \item използва се рядко
  \item Пример: \lstinline{a = (cout << x, x);}
  \end{itemize}
\end{frame}

\begin{frame}
  \frametitle{Съкратени оператори за присвояване}

  \begin{itemize}[<+->]
  \item \tt{a = a + 2} $\Leftrightarrow$ \tt{a += 2}
  \item \tt{-=}, \tt{*=}, \tt{/=}, \tt{\%=}
  \item \tt{a = a + 1}  $\Leftrightarrow$ \tt{++a}
  \item \tt{a = a - 1}  $\Leftrightarrow$ \tt{--a}
  \item \tt{a++} увеличава a с 1, но връща предишната стойност на \tt a
    \begin{itemize}
    \item \tt{a++} $\Leftrightarrow$ \tt{(a = (tmp = a) + 1, tmp)}
    \end{itemize}
  \item \tt{a--} действа аналогично
  \end{itemize}
\end{frame}

\begin{frame}[fragile]
  \frametitle{Оператор за блок}

  \begin{itemize}[<+->]
  \item \tta\{ \{ <оператор> \} \tta\}
  \item \tta\{ <оператор$_1$> <оператор$_2$> \ldots <оператор$_n$> \tta\}
  \item Вложени блокове\\
\begin{lstlisting}
{
  int x = 2;
  {
    x += 2;
    cout << x;
  }
}
\end{lstlisting}
  \end{itemize}
\end{frame}

\begin{frame}
  \frametitle{Област на действие (scope)}
  
  \begin{itemize}[<+->]
  \item областта на действие се простира от дефиницията на
    променливата до края на блока, в който е дефинирана
  \item дефиниция на променлива със същото име в същия блок е
    забранена
  \item дефиниция на променлива във вложен блок покрива всички външни
    дефиниции със същото име
  \end{itemize}
  %TBD: пример
\end{frame}

\begin{frame}
  \frametitle{Празен оператор}
  \begin{itemize}
  \item \tta;
  \item \tt; $\Leftrightarrow$ \tt{\{\}}
  \end{itemize}
\end{frame}

\begin{frame}
  \frametitle{Условен оператор}
  
  \begin{itemize}[<+->]
  \item \tta{if (}<израз>\tta) <оператор> [\tta{else} <оператор>]
  \item \lst{if ( x < 2 ) y = 2;} $\Leftrightarrow$ \lst{if ( x > 5 ) y = 5; else y = 3;}
  \item \lst{if (A) X;} $\Leftrightarrow$ \lst{if (A) X; else;}
  \item \lst{if (!A) X; else Y;} $\Leftrightarrow$ \lst{if (A) Y; else X;}
  \item \lst{if (A && B) X; else Y;} $\Leftrightarrow$ \lst{if (A) if (B) X; else Y; else Y;}
  \item \lst{if (A || B) X; else Y;} $\Leftrightarrow$ \lst{if (A) X; else if (B) X; else Y;}
  \end{itemize}
\end{frame}

\begin{frame}
  \frametitle{Условна \alert{операция}}
  \begin{itemize}[<+->]
  \item{} <булев\_израз> ? <израз> : <израз>
  \item триместна (тернарна) операция
  \item Пример: \lst{x = (y < 2) ? y + 1 : y - 2;}
  \item \lst{A} $\Leftrightarrow$ \lst{A ? true : false}
  \item \lst{!A} $\Leftrightarrow$ \lst{A ? false : true}
  \item \lst{A && B} $\Leftrightarrow$ \lst{A ? B : false}
  \item \lst{A || B} $\Leftrightarrow$ \lst{A ? true : B}
  \end{itemize}
\end{frame}

\begin{frame}
  \frametitle{Задачи за условен оператор}
  
  \begin{enumerate}
  \item Да се провери дали три числа образуват растяща редица
  \item Да се намери най-малкото от три числа
  \item Да се подредят три числа в растяща редица
  \item Да се провери дали три числа образуват Питагорова тройка
  \end{enumerate}
\end{frame}

\begin{frame}[fragile]
  \frametitle{Оператор за многозначен избор}
  
  \begin{itemize}
  \item \tta{switch (}<израз>\tta{) \{}\\
	{ \tta{case }<константен\_израз> \tta: { <оператор> } }\\
	[ \tta{default :} { <оператор> } ]\\
      \tta\}
  \item Пример:
\begin{lstlisting}
switch (x) {
	case 1 : x++;
	case 2 : x += 2;
	default : x += 5;
}
\end{lstlisting}
  \end{itemize}
\end{frame}

\begin{frame}[fragile]
  \frametitle{Оператор за прекъсване}
  
  \begin{itemize}
  \item \tta{break;}
  \item 
\begin{lstlisting}
switch (x) {
	case 1 : x++; break;
	case 2 : x += 2; break;
	default : x += 5;
}
\end{lstlisting}
  \end{itemize}
\end{frame}

\begin{frame}
  \frametitle{Задачи за многозначен избор}
  
  \begin{enumerate}
  \item Да се пресметне избрана от потребителя целочислена аритметична операция
  \item Да се провери дали дадена буква е гласна или съгласна
  \end{enumerate}
\end{frame}
\end{document}
