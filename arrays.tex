\documentclass{beamer}
\usepackage{up}

\title{Масиви и низове}

\date{9 ноември 2016 г.}


\begin{document}

\begin{frame}
  \titlepage
\end{frame}

\section{Масиви}

\begin{frame}
  \frametitle{Логическо описание}

  Масивът
  \begin{itemize}
  \item е съставен тип данни
  \item представя крайни редици от елементи
  \item всички елементи са от един и същи тип
  \item позволява произволен достъп до всеки негов елемент по номер (индекс)
  \end{itemize}
\end{frame}

\begin{frame}
  \frametitle{Дефиниция на масив}

  <тип> <идентификатор> \tta[ [<константа] \tta]\\
  \hspace{5ex} [ \tta= \tta\{ <константа> \{\tta, <константа> \} \tta\} ] \tta;\\[1em]
  \pause
  \textbf{Примери:}
  \begin{itemize}[<+->]
  \item \lst{bool b[10];}
  \item \lst{double x[3] = \{ 0.5, 1.5, 2.5 \}, y = 3.8;}
  \item \lst{int a[] = \{ 3 + 2, 2 * 4 \};} \eqv\ \lst{int a[2] = \{ 5, 8 \};}
  \item \lst{float f[4] = \{ 2.3, 4.5 \};} \eqv\ \lst{float f[4] = \{ 2.3, 4.5, 0, 0 \};}
  \end{itemize}
\end{frame}

\begin{frame}
  \frametitle{Физическо представяне}

  \scriptsize
  \begin{tabular}{*{11}{|c}|}
    \multicolumn{11}{l}{\tt a}\\
    \hline
    \rowcolor{diagramblue}
    \tt{a[0]}&\tt{a[1]}&\tt{a[2]}&\tt{a[3]}&\tt{a[4]}&\tt{a[5]}&\tt{a[6]}&\tt{a[7]}&\tt{a[8]}&\tt{a[9]}&\tt{a[10]}\\
    \hline
  \end{tabular}
\end{frame}

\begin{frame}
  \frametitle{Операции за работа с масиви}

  \begin{itemize}[<+->]
  \item Достъп до елемент по индекс: <масив>\tta[<цяло\_число>\tta]
  \item \textbf{Примери}:
    \begin{itemize}
    \item \lst{x = a[2];} (\alert{rvalue})
    \item \lst{a[i] = 7;} (\alert{lvalue!})
    \item \alert{Внимание: няма проверка за коректност!}
    \end{itemize}
  \item Няма присвояване
    \begin{itemize}
    \item \sta{a = b}
    \end{itemize}
  \item Няма поелементно сравнение
    \begin{itemize}
    \item \lst{a == b} винаги връща \lst{false} ако \tta a и \tta b са различни масиви, дори и да имат еднакви елементи
    \end{itemize}
  \item Няма операции за вход и изход
    \begin{itemize}
    \item \sta{cin >{}> a;}
    \item \lst{cout << a;} извежда адреса на \tt{a}
    \end{itemize}
  \end{itemize}
\end{frame}

\begin{frame}
  \frametitle{Задачи за масиви}

  \begin{itemize}[<+->]
    \item Да се въведе масив от числа
    \item Да се изведе масив от числа
    \item Да се намери сумата на числата в даден масив
    \item Да се провери дали дадено число се среща в масив
    \item Да се провери дали числата в масив нарастват монотонно
    \item Да се провери дали всички числа в даден масив са различни
    \item Да се подредят числата в даден масив в нарастващ ред
    \item Да се слеят два масива подредени в нарастващ ред
  \end{itemize}
\end{frame}

\section{Низове}

\begin{frame}
  \frametitle{Низове: описание и представяне}

  \begin{itemize}[<+->]
  \item \textbf{Описание:} \alert{Низ} наричаме последователност от символи
    \begin{itemize}
    \item последователност от 0 символи наричаме \alert{празен низ}
    \end{itemize}
  \item \textbf{Представяне в C++:} Масив от символи (\lst{char}), в който след последния символ в низа е записан \alert{терминиращия символ} \term
    \begin{itemize}
    \item \term\ е първият символ в ASCII таблицата, с код 0
    \end{itemize}
  \item \textbf{Примери:}
    \begin{itemize}
    \item \lst{char word[] = \{ 'H', 'e', 'l', 'l', 'o', '\\0' \};}
    \item \lst{char word[6] = \{ 'H', 'e', 'l', 'l', 'o' \};}
    \item \lst{char word[100] = \"Hello\";}
    \item \sta{char word[5] = \quot Hello\quot;}
    \item \lst{char word[6] = \"Hello\";}
    \item \sta{char word[] = \quot Hello\quot;}
    \end{itemize}
  \end{itemize}
\end{frame}

\begin{frame}
  \frametitle{Операции за работа с низове}

  \begin{itemize}[<+->]
  \item Вход (\tta{>{}>}, \tta{cin.getline(}<низ>\tta))
  \item Изход (\tta{<{}<})
  \item Индексиране (\tta{[]})
  \item Няма присвоявание! (\sta{a = b})
  \item Няма поелементно сравнение! (\sta{a == b})
  \item но...
  \item ...има вградени функции!
  \end{itemize}
\end{frame}

\begin{frame}
  \frametitle{Вградени функции за низове}

  \lst{\#include <cstring>}

  \begin{itemize}[<+->]
  \item \tta{strlen(}<низ>\tta)
    \begin{itemize}
    \item връща дължината на <низ>, т.е. броя символи без \term
    \end{itemize}
  \item \tta{strcpy(}<буфер>, <низ>\tta)
    \begin{itemize}
    \item прехвърля всички символи от <низ> в <буфер>
    \item връща <буфер>
    \item \alert{отговорност на програмиста е да осигури, че в <буфер> да има достатъчно място да поеме всички символи на <низ>}
    \end{itemize}
  \item \tta{strcmp(}<низ$_1$>, <низ$_2$>\tta)
    \begin{itemize}
    \item сравнява два низа \alert{лексикорафски} (речникова наредба)
    \item връща -1, ако <низ$_1$> е преди <низ$_2$>
    \item връща 0, ако <низ$_1$> съвпада с <низ$_2$>
    \item връща 1, ако <низ$_1$> е след <низ$_2$>
    \item \textbf{Интуиция:} ``знакът'' на ``разликата'' <низ$_1$> $-$ <низ$_2$>
    \item \textbf{Свойство:} \lst{strcmp(s1, s2) == -strcmp(s2, s1)}
    \end{itemize}
  \end{itemize}
\end{frame}

\begin{frame}
  \frametitle{Вградени функции за низове}

  \begin{itemize}[<+->]
  \item \tta{strcat(}<низ$_1$>, <низ$_2$>\tta)
    \begin{itemize}
    \item \alert{конкатенация} (слепване) на низове
    \item записва символите на <низ$_2$> в края на <низ$_1$>
    \item старият терминиращ символ се изтрива и се записва нов
    \item връща <низ$_1$>
    \item \alert{отговорност на програмиста е да осигури, че в <низ$_1$> да има достатъчно място да поеме всички символи на <низ$_2$>}
    \end{itemize}
  \item \tta{strchr(}<низ>, <символ>\tta)
    \begin{itemize}
    \item търсене на <символ> в <низ>
    \item връща \alert{суфикса} на <низ> от първото срещане на <символ>
    \item връща 0, ако <символ> не се среща в <низ>
    \end{itemize}
  \item \tta{strstr(}<низ>, <подниз>\tta)
    \begin{itemize}
    \item търсене на <подниз> в <низ>
    \item т.е. всички символи на <подниз> да се срещат последователно в <низ>
    \item връща \alert{суфикса} на <низ> от първото срещане на <подниз>
    \item връща 0, ако <символ> не се среща в <низ>
    \end{itemize}
  \end{itemize}
\end{frame}

\begin{frame}
  \frametitle{Задачи за низове}

  \begin{itemize}[<+->]
  \item Да се провери дали даден низ е \alert{палиндром}
    \begin{itemize}
    \item чете се еднакво в двете посоки
    \item \lst{\"abba\"}, \lst{\"racecar\"}, \lst{\"risetovotesir\"}, \lst{\"whasitacaroracatisaw\"}
    \end{itemize}
  \item Да се преброят думите в даден низ
    \begin{itemize}
    \item Считаме, че за разделители служат всички символи, които не са букви.
    \end{itemize}
  \item Да се пресметне аритметичен израз, записан в низ
    \begin{itemize}
    \item{} <израз> ::= <число>\{<операция><число>\}\tta=
    \item{} <число> ::= <цифра>\{<цифра>\}
    \item{} <операция> ::= \tta+ | \tta- | \tta* | \tta/ | \tta\%
    \end{itemize}
  \end{itemize}
\end{frame}

\begin{frame}
  \frametitle{Проблеми при работа с низове}

  \begin{itemize}[<+->]
  \item Излизане извън буфера (buffer overflow)
    \begin{itemize}
    \item \sta{char a[10] = \quot Hello, world!\quot}
    \item \lst{char b[] = \"Hello,\", c[] = \" world!\";}
    \item \sta{strcat(b, c);}
    \item \sta{strcpy(b, c);}
    \end{itemize}
  \item Нетерминирани низове
    \begin{itemize}
    \item \lst{char word[5] = \{ 'H', 'e', 'l', 'l', 'o' \};}
    \item \sta{cout <{}< strlen(a);}
    \item \lst{char b[10];}
    \item \sta{strcpy(b, a);}
    \end{itemize}
  \end{itemize}
\end{frame}

\end{document}
