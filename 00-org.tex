\documentclass[alsotrans]{beamerswitch}
\usepackage{iprog}

\title{Организация на курса}

\date{4 октомври 2021 г.}

\begin{document}

\begin{frame}
  \titlepage
\end{frame}

\begin{frame}
  \frametitle{Екип}

  \begin{itemize}
  \item \textbf{1 група семинар:} Теодор Тошков
  \item \textbf{1 група практикум:} Димитрина Василева, Пламен Динев
  \item \textbf{2 група семинар:} Иф Антонов
  \item \textbf{2 група практикум:} Александър Димитров, Мартин Дачев
  \item \textbf{3 и 4 група семинар:} гл. ас. д-р Дафина Петкова
  \item \textbf{3 група практикум:} Валентин Димитров, Димо Чанев
  \item \textbf{4 група практикум:} Антонио Георгиев, Борислав Карапанов
  \end{itemize}
\end{frame}

\begin{frame}
  \frametitle{Схема за оценяване}

  \begin{itemize}
  \item Обобщено представяне на студентите от семестъра (ОПСС)
    \begin{itemize}
    \item Домашни работи: 2 бр. (Д$_{1,2}$)
    \item Контролни работи: 2 бр. (K$_{1,2}$)
    \item Бонус: обратна връзка от работа през семестъра ($0 \leq \text{ОВ} \leq 0{,}75$)\\
      \begin{equation*}
        \text{ОПСС} = \frac{\text{Д}_1 + \text{Д}_2 + \text{K}_1 + \text{K}_2}4 + \text{ОВ} 
      \end{equation*}
    \end{itemize}
  \item Изпити (И)
    \begin{itemize}
    \item Писмен изпит (ПИ)
    \item Теоретичен изпит (ТИ)\\[-7.5ex]
      \begin{equation*}
        \qquad\qquad\text{И} = \frac{\text{ПИ} + \text{ТИ}}2
      \end{equation*}
    \end{itemize}
  \end{itemize}
  \vspace{2ex}
  \begin{equation*}
    \text{Крайна оценка} = \left[ \frac{\text{ОПСС} + \text{И}}2 \right]
  \end{equation*}
\end{frame}

\begin{frame}
  \frametitle{Канали за комуникация}

  \begin{itemize}
  \item Learn.fmi: \url{https://learn.fmi.uni-sofia.bg/}
  \item Бакалаври, зимен семестър 2021/2022
    \begin{itemize}
    \item КН
      \begin{itemize}
      \item Увод в програмирането (КН) 2021/22
      \end{itemize}
    \end{itemize}
  \item Онлайн излъчване с Google Meet: \url{https://meet.google.com/ice-rbgv-npw}
  \item Discord сървър на ФМИ: \url{https://discord.com/invite/3ygScbuqZt}
    \begin{itemize}
    \item \tt{\#уп-кн-студенти}
    \end{itemize}
  \end{itemize}
\end{frame}
\end{document}
