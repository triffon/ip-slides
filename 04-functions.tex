\documentclass[alsotrans]{beamerswitch}
\usepackage{iprog}

\title{Функции}
\subtitle{(част 1)}

\date{8--15 ноември 2021 г.}

\newcommand{\RR}{\mathbb{R}}

\begin{document}

\begin{frame}
  \titlepage
\end{frame}

\section{Основни понятия}

\begin{frame}
  \frametitle{Функциите в математиката}

  \begin{itemize}[<+->]
  \item Какво е функция в математиката?
  \item $f : Dom \to Ran$
    \begin{itemize}
    \item $Dom$ --- дефиниционна област
    \item $Ran$ --- обхват, област от стойности
    \end{itemize}
  \item Нека $F \subseteq Dom \times Ran$
  \item така че $\forall x \exists ! y (x,y) \in F$ (функционалност)
  \item еднозначност: ако $(x,y_1) \in F$ и $(x,y_2) \in F$, тогава $y_1 = y_2$
  \item Ако $(x,y)\in F$ пишем $f(x) = y$.
  \item $F$ --- графика на функцията $f$
  \end{itemize}
\end{frame}

\begin{frame}[fragile]
  \frametitle{Какво е програмна функция?}

  \begin{itemize}[<+->]
  \item Относително независима част от програмата, извършваща определено пресмятане
  \item Може да бъде използвана многократно
  \item \textbf{Пример 1}
    \begin{itemize}
    \item $f:\RR \to \RR$
    \item $f(x) = x^2$
    \item \lstinline!double square(double x) { return x * x; }!
    \end{itemize}
  \item \textbf{Пример 2}
    \begin{itemize}
    \item $d:\RR\times\RR\times\RR\times\RR \to \RR$
    \item $d(x_1,y_1,x_2,y_2) = \sqrt{(x_2-x_1)^2 + (y_2-y_1)^2}$
    \item
\begin{lstlisting}
double distance(double x1, double y1,
                double x2, double y2) {
  return sqrt(square(x2 - x1) + square(y2 - y1));
}
\end{lstlisting}
    \end{itemize}
  \end{itemize}
\end{frame}

\begin{frame}[fragile]
  \frametitle{Какво е подпрограма?}

  \begin{itemize}[<+->]
  \item Относително независима част от програмата, извършваща определена последователност от действия
  \item Може да бъде изпълнявана многократно
  \item Още: процедура, метод
  \item \textbf{Примери:}
\begin{lstlisting}
void printHello() {
  cout << "Hello!\n";
}
\end{lstlisting}
\onslide<+->
\begin{lstlisting}
void printReverseDigits(int n) {
  while (n > 0) {
    cout << n % 10;
    n /= 10;
  }
}
\end{lstlisting}
\end{itemize}
\end{frame}

\begin{frame}[fragile]
  \frametitle{Процедури и функции}

  \begin{itemize}[<+->]
  \item Функцията извършва пресмятане и връща резултат
  \item Процедурата изпълнява поредица от оператори
  \item Понякога двете понятия се смесват...
\begin{lstlisting}
int readNumber(int from, int to) {
  int n;
  do {
    cout << "n = "; cin >> n;
  } while (n < from || n > to);
  return n;
}
\end{lstlisting}
  \item В C++ се наричат просто ``функции''
  \end{itemize}
\end{frame}

\begin{frame}
  \frametitle{Дефиниране на функция}

  \begin{itemize}[<+->]
  \item{} <сигнатура> \tta\{ <тяло> \tta\}
  \item{} <сигнатура> ::= [ <тип\_резултат> | \tta{void} ]\\
    <идентификатор> \tta( <\textbf{формални}\_параметри> \tta)
    \begin{itemize}
    \item \tta{void} = празен тип, не връща резултат
    \item ако типът на резултата се пропусне, подразбира се \tt{int}
    \end{itemize}
  \item{} <\textbf{формални}\_параметри> ::=  \\
    <празно> | \tta{void} | <параметър> \{\tta, <параметър> \}
  \item{} <параметър> ::= <тип> [<идентификатор>]
    \begin{itemize}
    \item ако <идентификатор> се пропусне, параметърът няма име и не се използва
    \item \exa $f(x,y) = x + 5$
    \end{itemize}
  \item{} <тяло> ::= \tta\{ <оператор> \tta\}
  \end{itemize}
\end{frame}

\begin{frame}
  \frametitle{Извикване на функция}

  \begin{itemize}[<+->]
  \item{} <име>\tta(<\textbf{фактически}\_параметри>\tta)
  \item{} <\textbf{фактически}\_параметри> ::= \\
    <празно> | \tta{void} | <израз> \{\tta, <израз> \}
  \item извикването на функция всъщност е \alert{операция} с много висок приоритет
  \item типът на фактическия параметър се съпоставя с типа на съответния формален параметър
    \begin{itemize}
    \item ако се налага, прави се преобразуване на типовете
    \item{} <тип> <формален\_параметър> = <фактически\_параметър>
    \end{itemize}
  \end{itemize}
\end{frame}

\begin{frame}
  \frametitle{Връщане на резултат}

  \begin{itemize}[<+->]
  \item \tta{return} [<израз>]\tta;
  \item оператор за връщане на резултат на функция
  \item типът на <израз> се съпоставя с типа на резултата на функцията
    \begin{itemize}
    \item ако се налага, прави се преобразуване на типовете
    \end{itemize}
  \item работата на функцията се прекратява незабавно
  \item стойността на <израз> е резултатът от извикването на функцията
  \end{itemize}
\end{frame}

\begin{frame}
  \frametitle{Деклариране на функция}

  \begin{itemize}[<+->]
  \item{} <декларация\_на\_функция> ::= <сигнатура>\tta;
  \item Декларацията е ``обещание'' за дефиниция на функция
  \item Декларацията не е задължителна
  \item Една функция може да бъде декларирана няколко пъти...
  \item ...но може да бъде дефинирана \alert{само веднъж}
  \item Неизпълнените обещания водят до проблеми...
    \begin{itemize}
    \item ...освен когато никой не разчита на тях
    \end{itemize}
  \end{itemize}
\end{frame}
\end{document}
