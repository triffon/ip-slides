\documentclass[alsotrans]{beamerswitch}
\usepackage{up}

\title{Функции}

\subtitle{(част 3)}

\date[20.12.2018--3.01.2019]{20 декември 2018 г. --- 3 януари 2019 г.}

\usetikzlibrary{arrows.meta}

\begin{document}

\begin{frame}
  \titlepage
\end{frame}

\section{Програмен стек}

\begin{frame}
  \frametitle{Схема на програмната памет}

  \tikzset{
      m/.style={
      matrix of nodes,
      nodes={
        rectangle,
        minimum width=25ex,
        minimum height=2.5em
        }
    }
  }

  \begin{center}
    \begin{tikzpicture}
      \matrix (a) [m] {
        \ldots \\
        |[minimum height=5em,fill=diagramnavy]| Програмен стек \\
        |[fill=diagramnavy]| Статични данни \\
        |[fill=diagramblue]| Програмен код \\
        \ldots \\
      };
      \foreach\x in {east,west} { \draw (a-1-1.north \x) -- (a-5-1.south \x); }
      \foreach\x in {2,...,5} { \draw (a-\x-1.north west) -- (a-\x-1.north east); }
    \end{tikzpicture}
  \end{center}
\end{frame}

\begin{frame}
  \frametitle{Програмен стек}

  \tikzset{
      l/.style={-{Stealth[length=1.5ex]}},
      m/.style={
      matrix of nodes,
      nodes={
        rectangle,
        minimum width=25ex,
        minimum height=2em,
        fill=diagramblue,
        draw=black
      },
      nodes in empty cells
    }
  }

  \begin{center}
    \begin{tikzpicture}
        \matrix (a) [m] {
          |(sborder) [minimum height = 8em]| \ldots \\
          |(stop)|
          данни \\
          данни \\
          данни \\
          |(sbottom)|
          данни \\
        };
        \arrowto{sbottom.south west}{дъно на стека}{above left}
        \arrowto{stop.north east}{връх на стека}{below right}
        \arrowto{sborder.north east}{граница на стека}{below right}
    \end{tikzpicture}
  \end{center}
\end{frame}

\begin{frame}
  \frametitle{Програмен стек}

  \tikzset{
      l/.style={-{Stealth[length=1.5ex]}},
      m/.style={
      matrix of nodes,
      nodes={
        rectangle,
        minimum width=25ex,
        minimum height=2em,
        fill=diagramblue,
        draw=black
        }
    }
  }

  \begin{center}
    \begin{tikzpicture}
        \matrix (a) [m] {
          |(sborder) [minimum height = 6em]| \ldots \\
          |(stop)|
          данни \\
          данни \\
          данни \\
          данни \\
          |(sbottom)|
          данни \\
        };
        \arrowto{sbottom.south west}{дъно на стека}{above left}
        \arrowto{stop.north east}{връх на стека}{below right}
        \arrowto{sborder.north east}{граница на стека}{below right}
    \end{tikzpicture}
  \end{center}
\end{frame}

\begin{frame}
  \frametitle{Програмен стек}

  \tikzset{
      l/.style={-{Stealth[length=1.5ex]}},
      m/.style={
      matrix of nodes,
      nodes={
        rectangle,
        minimum width=25ex,
        minimum height=2em,
        fill=diagramblue,
        draw=black
        }
    }
  }

  % TODO: анимация на стека
  \begin{center}
    \begin{tikzpicture}
        \matrix (a) [m] {
          |(sborder) [minimum height = 12em]| \ldots \\
          |(stop)|
          данни \\
          |(sbottom)|
          данни \\
        };
        \arrowto{sbottom.south west}{дъно на стека}{above left}
        \arrowto{stop.north east}{връх на стека}{below right}
        \arrowto{sborder.north east}{граница на стека}{below right}
    \end{tikzpicture}
  \end{center}
\end{frame}


\begin{frame}
  \frametitle{Стекова рамка на функция}

  \tikzset{
      l/.style={-{Stealth[length=1.5ex]}},
      m/.style={
      matrix of nodes,
      nodes={
        rectangle,
        minimum width=25ex,
        minimum height=2em,
        fill=diagramblue,
        draw=black
        }
      },
      dots/.style={minimum height=5em}
  }

  \begin{center}
    \begin{tikzpicture}
      \matrix (a) [m] {
        |[dots]| \ldots \\
        локални променливи \\
        адрес за връщане \\
        фактически параметри \\
        |[dots]|  \ldots \\
      };
      \draw[very thick] (a-2-1.north west) rectangle (a-4-1.south east);
      \arrowto{a-2-1.south east}{рамков указател}{below right}
    \end{tikzpicture}
  \end{center}
\end{frame}

\begin{frame}
  \frametitle{Област за програмен код}

  \tikzset{
    l/.style={-{Stealth[length=1.5ex]}},
    m/.style={
      matrix of nodes,
      nodes={
        rectangle,
        minimum width=30ex,
        minimum height=2em,
        fill=diagramblue,
        draw=black
      }},
    dots/.style={minimum height=5em}
  }

  \begin{center}
    \begin{tikzpicture}
      \matrix (a) [m] {
        |[dots]| \ldots \\
        \tt{square} \\
        \tt{heron} \\
        |(t)| \tt{calculateTriangleArea} \\
        \tt{distance} \\
        \tt{main} \\
      };
      \arrowto{t.east}{програмен брояч}{below right}
    \end{tikzpicture}
  \end{center}
\end{frame}

\section{Предаване на параметри}

\begin{frame}
  \frametitle{Предаване по стойност (call by value)}

  \begin{itemize}[<+->]
  \item пресмята се стойността на фактическия параметър
  \item в стековата рамка на функцията се създава \alert{копие} на стойността
  \item всяка промяна на стойността остава локална за функцията
  \item при завършване на функцията, предадената стойност и всички промени над нея \alert{изчезват}
  \end{itemize}
\end{frame}

\begin{frame}
  \frametitle{Предаване с препратка (call by reference)}

  \begin{itemize}[<+->]
  \item Понякога искаме промените във \textbf{формалните} параметри
    да се отразят във \textbf{фактическите} параметри
  \item Тогава трябва да обявим, че искаме фактическите параметри да могат да бъдат променяни
  \item{} <параметър> ::= <тип>\tta{\&} <идентификатор>
  \item \textbf{Примери:}
    \begin{itemize}
    \item \lst{int add5(int\& x) \{ x += 5; return x; \}}
    \item \alert{фактическият параметър трябва да е lvalue!}
    \item \sta{add5(3);}
    \item \lst{int a = 3; cout << add5(a) << ' ' << a;}
    \end{itemize}
  \end{itemize}
\end{frame}

\begin{frame}[fragile]
  \frametitle{Пример за предаване с препратка}

  Размяна на две променливи
\begin{lstlisting}
void swap(int& x, int& y) {
  int tmp = x;
  x = y;
  y = tmp;
}
\end{lstlisting}
\pause
\begin{lstlisting}
int main() {
  int a = 5, b = 8;
  swap(a, b);
  cout << a << ' ' << b << endl;
}
\end{lstlisting}
\end{frame}

\begin{frame}
  \frametitle{Стекова рамка при предаване с препратки}

  \tikzset{
    m/.style={
      matrix of nodes,
      nodes={
        text height=1ex,
        text depth=0.25ex,
        minimum height=1em,
        minimum width=8ex
      },
      nodes in empty cells,
      ampersand replacement=\&
    },
    dots/.style={minimum height=5em},
    table/.style={
      rectangle,
      very thin,
      minimum width=25ex,
      fill=diagramblue,
      draw=black
    }
  }

  \begin{center}
    \begin{tikzpicture}
      \matrix [m] {
       |(swap)| \tt{swap} \& \tt{tmp} \& |(tmp) [table]| \\
       \& \& |[table]| адрес на връщане \\
       \& \tt y \& |(y) [table]| \\
       \& \tt x \& |(x) [table]| \\
       |(main)| \tt{main} \& \tt b \& |(b) [table]|\\
       \& \tt a \& |(a) [table]| \\
       \& \& |(bottom) [table]| адрес на връщане \\
      };
      \draw[very thick] (swap.north west) -- (tmp.north west) rectangle (b.north east);
      \draw[very thick] (main.north west) -- (b.north west) rectangle (bottom.south east);
      \draw (y.east) to [bend left=45] (b.east);
      \draw (x.east) to [bend left=45] (a.east);
    \end{tikzpicture}
  \end{center}
\end{frame}

\begin{frame}
  \frametitle{Предаване по указател/адрес (call by pointer)}

  \begin{itemize}[<+->]
  \item Предава се \alert{адрес} вместо стойност
  \item Фактическите параметри трябва да са от тип ``указател към нещо''
  \item Функцията може да променя стойности на външни за функцията променливи \alert{през подадените ѝ указатели}
  \item \textbf{Примери:}
    \begin{itemize}
    \item \lst{int add5(int* px) \{ *px += 5; return *px; \}}
    \item \sta{add5(3);} \sta{add5(\&3);}
    \item \lst{int a = 3; cout << add5(&a) << ' ' << a;}
    \end{itemize}
  \end{itemize}
\end{frame}

\begin{frame}[fragile]
  \frametitle{Пример за предаване по указател}

  Размяна на две променливи
\begin{lstlisting}
void swap(int* p, int* q) {
  int tmp = *p;
  *p = *q;
  *q = tmp;
}
\end{lstlisting}
\pause
\begin{lstlisting}
int main() {
  int a = 5, b = 8;
  swap(&a, &b);
  cout << a << ' ' << b << endl;
}
\end{lstlisting}
\end{frame}

\begin{frame}
  \frametitle{Стекова рамка при предаване по указател}

  \tikzset{
    m/.style={
      matrix of nodes,
      nodes={
        text height=1ex,
        text depth=0.25ex,
        minimum height=1em,
        minimum width=8ex
      },
      nodes in empty cells,
      ampersand replacement=\&
    },
    dots/.style={minimum height=5em},
    table/.style={
      rectangle,
      very thin,
      minimum width=25ex,
      fill=diagramblue,
      draw=black
    }
  }

  \begin{center}
    \begin{tikzpicture}
      \matrix [m] {
       |(swap)| \tt{swap} \& \tt{tmp} \& |(tmp) [table]| \\
       \& \& |[table]| адрес на връщане \\
       \& \tt q \& |(q) [table]| \\
       \& \tt p \& |(p) [table]| \\
       |(main)| \tt{main} \& \tt b \& |(b) [table]|\\
       \& \tt a \& |(a) [table]| \\
       \& \& |(bottom) [table]| адрес на връщане \\
      };
      \draw[very thick] (swap.north west) -- (tmp.north west) rectangle (b.north east);
      \draw[very thick] (main.north west) -- (b.north west) rectangle (bottom.south east);
      \draw[->,>=Stealth] (q.east) to [bend left=60] (b.east);
      \draw[->,>=Stealth] (p.east) to [bend left=60] (a.east);
    \end{tikzpicture}
  \end{center}
\end{frame}

\begin{frame}
  \frametitle{Предаване на масиви като параметри}

  \small
  \begin{itemize}[<+->]
  \item{} <параметър\_масив> ::= <тип> <име> \tta[[<константен\_израз>]\tta] |\\
    \hspace{26ex}<тип>\tta* <име>
  \item всъщност...
  \item ...масивите се предават \alert{по указател}!
  \item ...затова размерът на масива в скобите се игнорира!
  \item ...затова промените в винаги се отразяват в оригинала!
  \end{itemize}
\end{frame}

\begin{frame}
  \frametitle{Предаване на многомерни масиви като параметри}

  \small
  \begin{itemize}[<+->]
  \item{} <параметър\_многомерен\_масив> ::= \\
    \hspace{10ex}<тип> <име> \tta[[<константа>]\tta]\{\tta[<константа>\tta]\} |\\
    \hspace{10ex}<тип> \tta{(*}<име>\tta)\{\tta[<константа\tta]\}
    \item многомерните масиви също се предават по указател
    \item първата размерност се игнорира
      \begin{itemize}
      \item останалите трябва да се укажат, за да работи правилно указателната аритметика
      \end{itemize}
    \item (поне) първата размерност трябва да се подава като параметър
    \item \alert{Внимание:} \lst{int* a[10]} \alert{е различно от}
      \lst{int (*a)[10]}!
      \begin{itemize}
      \item \lst{int* a[10]} \eqv масив от 10 указателя към цели числа
      \item \lst{int (*a)[10]} \eqv указател към масив от десет цели
        числа
      \item ...но понеже масивите от тип T могат да се разглеждат като
        указатели към тип T...
      \item \lst{int (*a)[10]} \eqv масив от масив от десет цели числа
      \item \lst{int (*a)[10]} \eqv двумерен масив от цели числа с 10
        колони
      \end{itemize}
  \end{itemize}
\end{frame}

\begin{frame}
  \frametitle{Примерни функции}

  \begin{enumerate}[<+->]
  \item Да се напише функция, която извежда матрица от числа
  \item Да се напише функция, която въвежда масив от низове
  \item Да се напише функция, която проверява дали дадена дума се съдържа в масив от низове
  \item Да се напише функция, която умножава две правоъгълни матрици
  \end{enumerate}
\end{frame}

\section{Връщане на резултати}

\begin{frame}[fragile]
  \frametitle{Указателите като върнат резултат}

  \textbf{Основно правило:} трябва да осигурим, че винаги връщаме указатели към обекти, които ще продължат да съществуват след като функцията приключи работа.\\[2em]
  \pause
  \textbf{Пример:}
\begin{lstlisting}
int* pointMax(int* p, int* q) {
  if (*p > *q)
    return p;
  return q;
}
...
int* r = pointMax(&a, &b); (*r)--;
\end{lstlisting}
\end{frame}

\begin{frame}[fragile]
  \frametitle{Препратките като върнат резултат}

  Ваши същото правило като за указателите: връщаме препратки към обекти, които ще останат ``живи''.\\[2em]
  \pause
  \textbf{Пример:}
\begin{lstlisting}
int& middle(int& x, int& y, int& z) {
  if (x <= y && y <= z || z <= y && y <= x)
    return y;
  if (y <= z && z <= x || x <= z && z <= y)
    return z;
  return x;
}
...
middle(a, b, c) = 5;
\end{lstlisting}
\end{frame}

\begin{frame}
  \frametitle{Масивите като върнат резултат}

  \begin{itemize}[<+->]
  \item Функциите \alert{не могат} да имат ``масив от T'' като тип на резултата
  \item ...но могат да имат тип на резултата ``указател към T''
  \item по този начин функциите могат да връщат като резултат \alert{едномерни масиви}
  \item \alert{Внимание:} връщат се само масиви, които ще продължат да съществуват след като функцията завърши
  \item \textbf{Примери:}
    \begin{itemize}
    \item Да се реализира \tt{strchr}
    \item Да се реализира \tt{strstr}
    \item Да се реализира функция, която връща позицията на първото различие между два низа
    \end{itemize}
  \end{itemize}
\end{frame}

\end{document}
