\documentclass[alsotrans]{beamerswitch}
\usepackage{up}

\title{Записи}

\usetikzlibrary{arrows.meta}
\usetikzlibrary{patterns}

\tikzset{
  l/.style={-{Stealth}},
  n/.style={
    rectangle,
    minimum width=3ex,
    minimum height=1em,
    text height=0.75em,
    text depth=0.25em,
    fill=diagramblue,
    draw},
  cell/.style={n,very thick},
  m/.style={
    matrix of nodes,
    ampersand replacement=\&,
    nodes=n,
    nodes in empty cells,
    column sep=-\pgflinewidth
  }
}

\date[17.01--20.02.2019 г.]{17 януари -- 20 февруари 2019 г.}

\begin{document}

\begin{frame}
  \titlepage
\end{frame}

\section{Тип данни запис}

% TODO: мотивиращ пример

\begin{frame}
  \frametitle{Логическо описание}

Записът е:
\begin{itemize}[<+->]
\item съставен тип данни
\item представя крайна редица от елементи
\item редицата е с \alert<6>{фиксирана дължина}
\item елементите могат да са от \alert<6>{различни типове}
\item произволен достъп до всеки елемент
\end{itemize}
\end{frame}

\begin{frame}[fragile]
  \frametitle{Дефиниция на запис}

  \begin{itemize}[<+->]
  \item \tta{struct} <име> \tta\{ <поле> \{ <поле> \} \tta{\};}
  \item{} <поле> ::= <тип> <идентификатор> \{\tta, <идентификатор> \}\tta;
  \item \textbf{Примери:}
  \item \lst@struct Point { double x, y; };@
  \item
\begin{lstlisting}
struct Student {
  char name[45];
  int fn;
  double grade;
};
\end{lstlisting}
  \end{itemize}
\end{frame}

\begin{frame}[fragile]
  \frametitle{Физическо представяне}

  \begin{center}
    \begin{onlyenv}<1-6>
      \begin{tikzpicture}
        \matrix [m] {
          \ldots \&
          |[minimum width=24ex,label={\tt{name}}] (name) | {}
          \&
          |[minimum width=8ex,label={\tt{fn}}] (fn) | {}
          \&
          |[minimum width=16ex,label={\tt{grade}}] (grade) | {}
          \& \ldots\\
        };
        \draw[very thick] (name.north west) rectangle (grade.south east);
      \end{tikzpicture}
    \end{onlyenv}
    \begin{onlyenv}<7->
      \begin{tikzpicture}
        \matrix [m] {
          \ldots \&
          |[minimum width=24ex,label={\tt{name}}] (name) | {}
          \&
          |[minimum width=2ex,preaction={fill, diagramblue}, pattern=crosshatch] (padding) | {}
          \&
          |[minimum width=8ex,label={\tt{fn}}] (fn) | {}
          \&
          |[minimum width=8ex,preaction={fill, diagramblue}, pattern=crosshatch] (padding) | {}
          \&
          |[minimum width=16ex,label={\tt{grade}}] (grade) | {}
          \& \ldots\\
        };
        \draw[very thick] (name.north west) rectangle (grade.south east);
      \end{tikzpicture}
    \end{onlyenv}
  \end{center}
  \begin{columns}[T,onlytextwidth]
    \begin{column}{.3\textwidth}
\begin{lstlisting}
struct Student {
  char name[45];
  int fn;
  double grade;
};
\end{lstlisting}
    \end{column}
    \begin{column}{.7\textwidth}
      \begin{itemize}[<+->]
      \item \lst{sizeof(S)} --- големина на структурата \tt S
      \item \lst{sizeof(Point)} = \alt<+->{16}?
      \item \lst{sizeof(Student)} = \alt<+->{\alert{64}}?
      \item \alert{Защо?}
      \item Полетата в структурите се подравняват до адрес кратен на големината им
      \item Улеснява обработката от процесора
      \end{itemize}
    \end{column}
  \end{columns}
\end{frame}

\begin{frame}[fragile]
  \frametitle{Променливи от тип запис}

  [\tta{struct}] <тип\_запис> <име> [ \tta{= \{} <израз> \{\tta, <израз> \} \tta\} ]\\
  \hspace{20ex} \{\tta, <име> [ \tta{= \{} <израз> \{\tta, <израз> \} \tta\} ] \}\tta;\\[2em]
  \pause
  \textbf{Примери:}
  \begin{itemize}[<+->]
  \item \lst!Point p1, p2 = { 1.2, 3.4 };!
  \item \lst!Student s1 = { "Иван Колев", 61234, 5.75 };!
  \end{itemize}
\end{frame}

\begin{frame}
  \frametitle{Операции над записи}

  \begin{itemize}[<+->]
  \item Присвояване (\tt=)
    \begin{itemize}
    \item Могат да се присвояват само структури от един и същи тип
    \item \lst{Point p3 = p1; p3 = p2;}
    \item \sta{Student s1 = p1;}
    \end{itemize}
  \item Достъп до поле (\tt.)
    \begin{itemize}
    \item{} <променлива>\tta.<име\_на\_поле>
    \item \lst{p1.x = 1.3; p2 = p1; p2.y = -p2.y;}
    \item \lst{s1.fn = 41000; cout << s1.grade;}
    \item \lst{cin.getline(s1.name, 45); s2 = s1;}
    \item \lst{int* p = &s1.fn;}
    \item \lst{char* s = s1.name;}
    \end{itemize}
  \item Няма операции за вход и изход
    \begin{itemize}
    \item \sta{cin >{}> s1;}
    \item \sta{cout <{}< p1;}
    \end{itemize}
  \end{itemize}
\end{frame}

\section{Съставни типове данни}

\begin{frame}[fragile]
  \frametitle{Масив от записи}

  Можем да комбинираме свободно съставните типове данни, за да създаваме произволно сложни потребителски типове данни.\\
  \pause
  \begin{itemize}[<+->]
  \item \lst!Student s[10] = { { "Петър Петров", 80000, 5.5},!\\
      \hspace{21ex}\lst!{ "Стефани Стефанова", 60000, 6 } };!
  \item \lst!strcpy(s[2].name, "Иван Иванов");!
  \item \lst!cout << s[1].fn;!
  \item \lst!for(int i = 0; i < n; i++)!\\
    \hspace{2ex}\lst!cin >> s[i].grade;!
  \end{itemize}
\end{frame}

\begin{frame}[fragile]
  \frametitle{Запис от записи}

\begin{lstlisting}
struct Team {
  Student s1, s2;
  char name[30];
};
\end{lstlisting}
\pause
  \begin{itemize}[<+->]
  \item \lst!Team team = { { "Диана", 80003, 5 },!\\
      \hspace{16ex}\lst!{ "Радослав", 60004, 6}, "Дислав"};!
  \item \lst!cout << team.name << ' ' << team.s2.name;!
  \item \lst!double teamGrade = (team.s1.grade + team.s2.grade) / 2;!
  \end{itemize}
\end{frame}

\begin{frame}
  \frametitle{Записи и функции}

  \begin{itemize}[<+->]
  \item Записите като параметри
    \begin{itemize}
    \item Предават се \alert{по стойност}, като простите типове данни
      \begin{itemize}
      \item за разлика от масивите!
      \end{itemize}
    \item промените във функциите са локални
    \end{itemize}
  \item Записите като върнат резултат
    \begin{itemize}
    \item Връщат се \alert{по стойност}, като простите типове данни
    \item Връща се копие на записа
    \end{itemize}
  \end{itemize}
\end{frame}

\begin{frame}
  \frametitle{Задачи за записи}

  \begin{enumerate}[<+->]
  \item Да се въведе масив от студенти
  \item Да се изведат студентите в таблица
  \item Да се намери средния успех на всички студенти
  \item Да се подредят студентите по Ф№
  \end{enumerate}
\end{frame}

\section{Рекурсивни типове данни}

\begin{frame}
  \frametitle{Указатели и препратки на записи}

  Можем да правим указатели и препратки на записите и техните полета

  \begin{itemize}[<+->]
  \item \lst{Student* ps1 = &s1, *ps2 = nullptr;}
  \item \lst{ps2 = ps1; *ps2 = s2;}
  \item \lst{Student& s3 = s1;}
  \item \lst{cout << s3.name;}
  \item \lst{s3 = s2;}
  \item Записите могат да се предават като параметри на функции по стойност, указател и препратка
  \end{itemize}
\end{frame}

\begin{frame}
  \frametitle{Достъп до поле на запис чрез указател}

  <указател\_към\_запис> \tta{->} <поле>\\[2em]
  \begin{itemize}[<+->]
  \item еквивалентно на \tta{(*}<указател\_към\_запис>\tta{).}<поле>
  \item \lst{ps1->grade += 0.5;}
  \item \lst{cout << ps2->fn;}
  \item \lst{Team* pteam = &team; cout << pteam->s1.name;}
  \end{itemize}
\end{frame}

\begin{frame}[fragile]
  \frametitle{Рекурсивни записи}

\begin{lstlisting}
struct Employee {
  char name[64];
  @\alert<4>{Employee boss;}@
};
\end{lstlisting}
  \pause
  \begin{itemize}[<+->]
  \item записът се дефинира чрез себе си
  \item \lst{sizeof(Employee)} = ?
  \item \alert{забранена рекурсия!}
  \end{itemize}
\end{frame}

\begin{frame}[fragile]
  \frametitle{Рекурсивни записи --- правилният начин}
\begin{lstlisting}
struct Employee {
  char name[64];
  Employee* boss;
};
\end{lstlisting}
  \pause
  \begin{itemize}[<+->]
  \item записът се дефинира чрез себе си...
  \item ... но съдържа \alert{указател към себе си}
  \item \lst{sizeof(Employee)} = \alt<+->{72}?
  \item \lst!Employee rector = { "Герджиков", nullptr },!\\
    \hspace{10ex}\lst!dean = { "Първанов", &rector },!\\
    \hspace{10ex}\lst!chair = { "Нишева", &dean },!\\
    \hspace{10ex}\lst!lector = { "Трифонов", &chair };!
  \item \lst{cout << lector.boss->boss->boss->name;}
  \end{itemize}
\end{frame}

\section{Абстракция със структури от данни}

\begin{frame}
  \frametitle{Абстракция със структури от данни}

  Записите позволяват дефиниране на потребителски типове данни и операции над тях.\\[1em]
  \pause
  \textbf{Пример:} Тип ``рационално число''
  \pause
  \begin{itemize}[<+->]
  \item \textbf{Логическо описание:} обикновена дроб
  \item \textbf{Физическо представяне:} запис с числител и знаменател
  \item \textbf{Базови операции:}
    \begin{itemize}[<.->]
    \item конструиране на рационално число
    \item получаване на числител
    \item получаване на знаменател
    \end{itemize}
  \item \textbf{Аритметични операции:}
    \begin{itemize}[<.->]
    \item събиране, изваждане
    \item умножение, деление
    \item сравнение
    \end{itemize}
  \item \textbf{Приложни програми}
  \end{itemize}
  \onslide<+->
  \textbf{Идея:} да изолираме вътрешното представяне на типа данни от операциите над него
\end{frame}

\begin{frame}
  \frametitle{Нива на абстракция}

  \renewcommand{\bua}{\bigg\uparrow}

  \begin{center}
    \begin{tabular}{|c|}
      \hline
      \textbf{приложни програми}\\
      \\
      \hline
      \multicolumn 1c\bua\\
      \hline
      \textbf{аритметични операции}\\
      (+, -, *, /, =, <, >, <=, >=)\\
      \hline
      \multicolumn 1c\bua\\
      \hline
      \textbf{базови операции}\\
      (конструктор, селектори за числител и знаменател)\\
      \hline
      \multicolumn 1c\bua\\
      \hline
      \textbf{физическо представяне}\\
      (запис от две цели числа)\\
      \hline
    \end{tabular}
  \end{center}
\end{frame}

\begin{frame}
  \frametitle{Обектно-ориентирано програмиране}

  \begin{itemize}
  \item структуриране на данните на концептуално ниво като ``обекти''
  \item обектите имат \textbf{свойства и методи} за работа с тях
  \item могат да се дефинират \textbf{нива на достъп} до компоненти на обекта
  \item представянето на обекта обикновено се \textbf{капсулира}
  \item еднотипни обекти се описват чрез \textbf{класове} от обекти
  \item възможностите на обект могат да се разширяват (\textbf{наследяване})
  \item един обект може да има различни проявления (\textbf{полиморфизъм})
  \end{itemize}
\end{frame}

\end{document}

\section*{Предстои}


\begin{frame}
  \begin{center}
    \Huge
    \alert{\textbf{ООП}}\\[0.5em]
    \Large
    февруари 2019\\[2em]
    \pause
    \small
    скоро на проектор около вас
  \end{center}
\end{frame}

\end{document}
